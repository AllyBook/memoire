\part{Aragon et Masson : Mémoires croisées}
\chapter{Rencontre d'anciens combattants}
    Pour refléter la relation des des hommes,  deux grands textes évoquent la relation entre Aragon et André Masson : le poème Cantate à André Masson par Aragon et le texte Salut [Louis Aragon]  de Masson. L’intérêt de ces deux textes n’est pas uniquement d’être écrit par les principaux intéressés, mais de revenir sur une amitié connue depuis le surréalisme, incontestable, évoquée ci-et-là dans des articles mais sans plus de commentaires développés sur la relation des deux hommes. Le lieu emblématique de la rue Blomet est fréquemment mentionné à ces occasions, puisque l’atelier de la rue Blomet est le symbole des rencontres des peintres et des poètes surréalistes, tels Masson et Aragon.  Et pourtant, l’un et l’autre expriment dans leur texte un rapport à l’autre au-delà de la forte amitié, en faisant un camarde à part des autres, inassimilable à d’autres visages. En outre, une autre grande particularité de ces textes est d’avoir été écrite tardivement, dans les dernières années de la vie respective de ces hommes de la même génération. De telle sorte que l’on peut se demander sis les deux cas, il s’agit de déclarer l’affection hors-norme que l’autre représente. Aragon confirme cette déclamation 	avec une brève phrase pour présenter sa cantate, qui fait office de préface à un livre d’images de Masson, « Préface abusive à dix images de l’amour ». Mais, dans cette déclaration volontairement lyrique avec son introduction par l’idée d’ « images d’amour » et de la forme de « cantate », Aragon parle d’abord du poids de sa propre longévité : « Et croyez moi je n’ai peur que de / Ne pas demain mourir encore » .
