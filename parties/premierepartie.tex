\chapter{Aragon et Masson : Mémoires croisées}
\section{Rencontre d'anciens combattants}
    Pour refléter la relation des des hommes,  deux grands textes évoquent la relation entre Aragon et André Masson : le poème \footcite[p681]{ecritssurla} \emph{Cantate à André Masson} par Aragon et le texte\emph{Salut [Louis Aragon]} de Masson. L’intérêt de ces deux textes n’est pas uniquement d’être écrit par les principaux intéressés, mais de revenir sur une amitié connue depuis le surréalisme, incontestable, évoquée ci-et-là dans des articles mais sans plus de commentaires développés sur la relation des deux hommes. Le lieu emblématique de la rue Blomet est fréquemment mentionné à ces occasions, puisque l’atelier de la rue Blomet est le symbole des rencontres des peintres et des poètes surréalistes, tels Masson et Aragon.  Et pourtant, l’un et l’autre expriment dans leur texte un rapport à l’autre au-delà de la forte amitié, en faisant un camarade à part des autres, inassimilable à d’autres visages. En outre, une autre grande particularité de ces textes est d’avoir été écrite tardivement, dans les dernières années de la vie respective de ces hommes de la même génération. De telle sorte que l’on peut se demander sis les deux cas, il s’agit de déclarer l’affection hors-norme que l’autre représente. Aragon confirme cette déclamation 	avec une brève phrase pour présenter sa cantate, qui fait office de préface à un livre d’images de Masson, « Préface abusive à dix images de l’amour ». Mais, dans cette déclaration volontairement lyrique avec son introduction par l’idée d’ « images d’amour » et de la forme de « cantate », Aragon parle d’abord du poids de sa propre longévité :\footcite[p681]{ecritssurla} « Et croyez moi je n’ai peur que de / Ne pas demain mourir encore » .

    C’est d’ailleurs un des grands points communs aux deux hommes, celle d’appartenir non seulement à la même génération mais de connaître une longévité plus importante que leurs proches partis avant eux. D’autant plus que l’un et l’autre ont connu les deux guerres et ont tous les deux combattus et subis dans leur démarche créatrice ultérieurement le traumatisme de la 1ère Guerre Mondiale. Leur rencontre, en 1923, marque dans l’histoire littéraire la transition en train de s’opérer entre cette partie groupe menée par Breton anciennement Dada en train d’aller vers le surréalisme. La rencontre n’est pas vécue dans le souvenir de Masson dans une relation d’égal à égal, mais comme un honneur pour lui de rencontrer un auteur qu’il avait lu et admirait :



\begin{quote} Ma rencontre avec Louis Aragon, ce fut par une belle matinée de printemps, en 1923, qu’eut lieu cet événement; par le truchement de Georges Limbour. Evénement, , j’insiste, car tel il fut : connaître l’auteur d’\emph{Anicet} ce n’était pas une mince faveur en ces années-là. C’était l’époque (le surréalisme encore dans les limbes) où un jeune, orienté au mieux, lisait \emph{tLittérature}, revue d’extrême pointe. Louis faisait partie de la scintillante pléiade; il était, au vrai, parmi les plus vifs animateurs, le plus combatif.\footcite[p84]{rebelle}\end{quote}

	Or, cette particularité que Masson relève dans les articles d’Aragon de \emph{Littérature}, est sans doute spécifiquement l’héritage retenu de Dada qui va le poursuivre encore pendant la période surréaliste, et même au-delà. La preuve en est avec \emph{Le traité du style}. Par ailleurs, cette énergie dans l’écriture d’Aragon que souligne André Masson, est aussi l’un des plus récurrents qualificatifs des critiques pour définir la propre esthétique de Masson : L’énergie serait donc à la fois un facteur d’attirance entre les deux hommes en même temps qu’un véritable processus de création. Il s’avère d’ailleurs que, comme par un jeu de miroirs, Aragon décrit aussi dans une strophe de sa contante cette même rencontre :

    
\begin{verse}    
Tout ce qui m’entoure aujourd’hui ressemble

A un grand dessin d’André Masson comme

Il y en eu plus d’un dans les premières

Années vingt Rue Blomet 

Un dessin qui semblait

Danser ses limites\footcite[p. 682]{ecritssurla} 
\end{verse}

	Ainsi, une analogie se révèle entre les premiers motifs d’admiration de l’un pour l’autre dans cette transition vers le surréalisme dans le courant des années vingt. Cette question des « limites » du trait, probablement les prémices du dessin automatique à venir dont Masson est l’un des emblèmes, peut également dans un sens plus large en terme de style qui croise l’idée de « vif » et « combattif » de Masson pour qualifier l’oeuvre d’Aragon. Cette énergie combative évoque d’une part en première évidence une attirance esthétique similaire chez l’écrivain et le peintre, mais rattachée à celle-ci une recherche philosophique infiniment liée : l’énergie furibonde dans le mouvement du trait chez Masson comme dans l’écriture provocatrice d’Aragon est sans doute le trait dans l’oeuvre des deux hommes qui connaît différentes évolutions mais demeure une caractéristique fondamentale. Il n’est pas anodin qu’Aragon âgé, et qui se met en scène dans son poème comme tel, revive sa jeunesse par l’intermédiaire du geste premier du trait de dessin, afin de relater justement une première rencontre avec le dessinateur. Par ses propos, Aragon confirme la certitude que fait Masson à propos de leur obsession commune sur un autre type de dépassement, celui du temps : 
\begin{verse}    
Et puis, et puis…sonnent les cinquante ans d’une harmonieuse amitié au cadran du temps - du temps absolu : celui des poètes er des peintres niant celui de l’horloge.\footcite[p84]{rebelle}\end{verse}


	Une telle philosophie ne peut qu’être le projet de toute une vie. C’est par cette phrase symbolique que Masson conclue son texte. Le rapport au temps est d’ailleurs un topos récurrent des \emph{Lettres françaises}. Particulièrement à partir de 1965, où le numéro du 4 au 10 novembre 1965 associe André Masson comme Aragon à la figure de la jeunesse. Cette petite phrase de Masson peut également laisser sous-entendre cette particularité de « l’harmonie » qu’aura été son amitié avec Aragon, c’est-à-dire insensible aux  évolutions du temps, et qui peut se concevoir comme une distinction à l’amitié plus mitigée que connaît Masson avec Breton, avec son départ du groupe surréaliste en 31, leurs retrouvailles au moment de la 2nd Guerre Mondiale et le départ pour l’Amérique, puis la rupture définitive de 1941. ce qui confirme l’hypothèse d’une vision commune sur l’oeuvre d’art, littéraire ou picturale, chez Aragon et Masson, puisque la question du temps traverse même traitée différemment les différentes orientations romanesques qui guident Aragon à travers les années.

	En outre, Aragon lui-même emploie pour désigner son souvenir de Masson jeune homme une de ses images les plus récurrentes, tant dans ses oeuvres poétiques et romanesques et qui confirme sa propre projection dans la personne du peintre André Masson : 

\begin{verse}
    
Et qui donc était dans la cour au fond

Ce peintre-miroir ses yeux noirs d’enfant

Portés sur la vie\footcite[p682]{ecritssurla}\end{verse}



	Avec la figure du miroir, Masson est désigné par Aragon comme le peintre de la vie. Ce qui est d’autant plus symbolique chez un peintre qui refuse le mimétique comme miroir du réel. Une vie  liée au concept de spontanéité dans le principe de traits jaillissants chez Masson et associés dans ce portrait à l’idée d’innocence. Masson est ainsi comparé à un petit garçon en pleine découverte de l’espace qui l’entoure. Un effet-miroir se manifeste d’ailleurs dans ces textes entre l’appel à la fête de Masson dans son texte sur Aragon, et cette confirmation du poète dans sa propre chantant avec son portrait qui présente son ami peintre comme un jouisseur de l’existence. Et pourtant, selon une conception plus proche de l’innocence dans l’imaginaire lié à l’enfance que d’une forme de débauche plus du côté de Sade, admiré pourtant de l’un et de l’autre. Mais on peut considérer que ces deux images ne sont pas incompatibles. On retrouve d’ailleurs dans les deux textes, un imaginaire de liberté totale se manifeste dans le lieu symbolique du café. La rue Blomet et le café sont les deux lieux de l’autre vie après la guerre. Plus précisément, cette autre vie est aussi la jeunesse, puisque c’est au café qu’Aragon se met en scène en train d’écrire son poème, en train de revivre au milieu des nouveaux jeunes gens : \enquote{ Survivre à quoi J’écris ceci / Dans un café de jeunes gens quelque part. » , « Dans un café de jeunes gens / Beaux comme les rencontres.}\footcite[p681]{ecritssurla} L’image du café chez Aragon représente donc symboliquement l’esprit festive sur laquelle s’arrête également Masson dans son propre texte. 

	 Mais, comme on peut s’y attendre devant le portrait du « peintre-miroir », c’est de sa propre enfance qu’Aragon finit par faire mention dans une thématique qui décrivait au départ André Masson comme le peintre éternellement jeune. Cette figure de l’enfance est d’ailleurs mêlée par les références intertextuelles liées au  poème de Victor Hugo, \emph{Booz endormi }: \enquote{ Car le jeune homme est beau, mais le vieillard est grand }, \enquote{Et l'on voit de la flamme aux yeux des jeunes gens, / Mais dans l’oeil du vieillard on voit de la lumière.}\footcite{hugo} En faisant d’André Masson le personnage de Booz lui-même, non seulement Aragon dépeint un portrait lyrique, mais il prête à Masson une forme d’aura qui le rapproche dans cette contante des personnages mythologiques peints par Masson lui-même. Des figures mythologiques , qui, par essence, échappent à toute temporalité :  \enquote{Booz, puisqu'il faut t’appeler de ce nom / Mythique.}\footcite[p685]{ecritssurla}. Désigner dans sa cantate André Masson comme Booz permet à Aragon de rendre hommage à Masson à la lumière de ses oeuvres mais aussi de ses qualités humaines, puisque ce sont ces dernières qui créent la force lumineuse, l’aura, de Booz. 

	Même le topos de la danse traverse à la fois le texte d’André Masson et la \emph{Cantate à André Masson }d’Aragon. Aragon, en évoquant le dessin \enquote{qui semblait / Danser ses limites}\footcite[p682]{ecritssurla}, Masson à propos d’une nostalgie des soirées surréalistes : 
 Peu après, l’aventure merveilleuse du surréalisme nous rapprocha davantage, à tel point, ô Nuits de Paris, que notre noctambulisme commun auquel se joignaient Michel Leiris et Georges Limbour était devenu notoire dans les milieux d’avant-garde. (Nous aimions la danse, la musique de jazz, la fête enfin…). Fête ! Locution populaire : “faire la vie“. Faire la fête était, pour nous, l’essai de faire de notre vie, une fête. 

	Le champ lexical, analysé par Masson lui-même dans son texte, dévoile un autre versant philosophique mode \emph{carpe diem} prêté au jeune groupe de surréalistes, et qui rappelle le besoin d’un retour aux festivités après les horreurs de la guerre. L’évocation de cette 1ère Guerre Mondiale qui suit cette ode à la fête démontre démontre dans ce principe de légèreté l’idée de se constituer une autre vie après le traumatisme. On retrouve dans le méta-discours de Masson les prémices du principe du jaillissement dans son oeuvre, l’un des traits de son style qui rendent ambigus la frontière entre figuration et abstraction. S’il refuse cette dernière, le jaillissement révèle le désir de concevoir l’oeuvre d’art comme une « fête ». De plus, à ce plaisir des festivités suit dans le paragraphe suivant l’évocation à son antithèse, la vie sur le champ de bataille avec la référence absolue du Chemin des Dames : 

\begin{quote}J’en reviens à \emph{Anicet}. Je ne savais pas à ce moment-là que son auteur en avait eu l’idée devant le Chemin des Dames (lieu si peu fait pour les dames et les demoiselles mais furieusement fréquenté par des hommes en masse et, cela, depuis les légions de César (j’en passe) jusqu’à celles de la Grande Armée). Je précédais Louis de quelques mois sur ce plateau redoutable. L’admirable c’est que nous sommes tous les deux rescapés de cette guerre (cette vieille guerre !) et que cela nous a permis de nous trouver à la terrasse d’un café proche de la place de Médicis. Nous étions encore dans nos jeunes années, mal essuyés des misères prodigieuses de la vie guerrière, mais nullement fanés, flétris, moroses, bien au contraire.\footcite[p85]{rebelle}\end{quote}

	Cette évocation à leur expérience commune de combattant est intéressante. D’abord, pour ce qu’elle ne mentionne pas, à savoir la fameuse blessure de guerre à la main d’André Masson, blessé au bras au Chemin des Dames. La guerre n’est d’ailleurs évoquée que par ses composantes, \enquote{un plateau redoutable}. Mais la guerre ne semble mentionnée que pour ramener, toujours avec le lieu symbolique du café, à l’extase de l’existence à laquelle aspirent tous ces jeunes gens. Toujours est-il que l’un des grands points communs d’Aragon et Masson dans leurs textes est de faire du café plus qu’un lieu mais le motif de leur rencontre, puisque que le café est désigné par les deux hommes comme le lieu d’expression de la vie, l’antithèse du Chemin des Dames. En cela, André Masson se rapproche tout comme un autre de ses très grands proches Michel Leiris de la théorie de la fête selon Caillois :

 \begin{quote}Pour cet auteur, en effet, la fêtée introduit une rupture du quotidien, dans le monde du travail, elle est aux jours ouvrables ce que le sacré est au profane, dans la mesure où elle oppose “une explosion intermittente à une terne continuité, une frénésie exaltante à la répétition quotidienne des mêmes préoccupations matérielles, le souffle puissant de l’effervescence commune aux calmes travaux où chacun s’affaire à l’écart, la concentration de la société à sa dispersion, la fièvre de ses instant culminants au tranquille labeur des phases atones de son existence“.\footcite[]{poitryguy}\end{quote}

	La dialectique qui régit la théorie de Caillois est reproduite dans le texte de  Masson entre l’existence pendant la guerre et les soirées surréalistes. Mais c’est par une multiplicité d’influences que Masson bâtit tout un concept autant esthétique que politique sur l’idée de fête, comme en atteste   son texte en homme à un autre de ses grands amis, Georges Bataille :
	
\begin{quote}
Le sacré, l’orgie, la notion de dépense - Dans un monde réduit aux seules obligations du travail, de la conscription, et autres servitudes, trouver la faille qui permettrait de s’épanouir à nouveau la Fête. Sans quoi une civilisation est boiteuse.\footcite[p74]{rebelle}
\end{quote}

	Or, faire voir la civilisation est toujours la ligne directrice d’une oeuvre de Masson, même dans son antithèse lors de la série de dessins Massacres. L’idée peut-être paradoxale, parce qu’elle part d’un concept inverse à celui des Philosophes des Lumières : Ces derniers puisaient comme valeur-phare d’une civilisation la raison. Mais Masson lie lui-même l’idée de fête à l’une de ses grandes références qu’est Nietzsche  et son apostrophe  : \enquote{Artistes, préparez-nous des fêtes !}\footcite[p39]{memoiremonde} et le summum que constitue cette notion dans le \emph{le Gai Savoir} de Nietzsche, \enquote{Qu’importe tout notre art dans les œuvres d’art, si l’art supérieur, qui est l’art des fêtes, se met à disparaître parmi nous !\footcite[]{nietzsche} }
	% Corriger page
	Masson, lui, part de la libération totale, de l’esprit et du corps. Cette tension est aussi celle que revendique Masson pour se définir, \enquote{Je suis un pessimiste gai.}\footcite[p. 8]{memoiremonde} La civilisation de Masson encourage cet apparent chaos parce qu’il manifeste avant tout l’expression libre et totale de l’homme. Mais en plus de cette philosophie, cette critique de l’abrutissement du travail désole déjà les réserves qu’omet Masson dans une lettre de 1935, et adressée à ce même Georges Bataille le 8 novembre 1935 :
	
\begin{quote}Cependant, je suis sûr que tout ce qui reposera sur le marxisme sera sordide, parce que cette doctrine ne repose que sur une idée fausse de l’homme. — L’homme pour moi est une réalité (en soi) - (J’exagère à dessein). Pour le marxiste l’homme n’est qu’une fonction (relative…à quoi ? au milieu !, un milieu fabriqué d’avance, sans réalité profonde.) Exemple : Je crois que ce n’est pas “l’autorité capitaliste“ qui abrutit l’ouvrier c’est \emph{l’Usine}. (Tu auras beau dire toi intellectuel, à l’ouvrier, qu’il est le sel de la terre il n’en restera pas moins un \emph{asservi à un travail contre nature}. […] — Ce ne sont pas les capitalistes (esclaves ou non) , qui nous “mènent à l’abîme“ mais bien les savants et les “inventeurs“ et en général toute manière rationaliste de considérer la vie.\footcite[p292]{anneessurrealistes}\end{quote}

	Une certaine similarité paraît émerger entre la théorie de Masson sur la \enquote{Fête} comme idéal philosophique parce que ‘elle repose sur l’homme, et ses réserves de 1935 sur le travail forcené qui réduit au contraire l’individu parce que la libération de l’esprit comme du corps ne peut pas exister. 

	D’autre part, si André Masson emploie l'idée de fête pour évoquer la période des soirées surréalistes, le métadiscours qu’il tient sur le mot élargit sa pensée, d’autant plus qu’il l’assimile à un verbe d’action, \enquote{faire la vie}. Sous-entendu : La vivre mais aussi la concevoir. Or, plusieurs de ses oeuvres intègrent dans leur titre le mot \enquote{fête }: \emph{Fête sanglante}, 1932, \emph{Une fête}, 1958, \emph{Fête galante chez les écorchés,} 1963, \emph{La fête des corps}, 1972. Comme pour Aragon et le vertige, la fête n’est pas vouée à durer, l’événement est éphémère. Et d’un autre côté, le terme le hante constamment dans ses oeuvres, tout comme Aragon malgré ses nouvelles réorientions d’écriture ne quitte pas l’obsession du vertige dans l’oeuvre romanesque comme poétique. Le rapport charnier au temps, plus particulièrement le pouvoir pour le créateur de dépasser les limites du temps, est ancré dans ces deux notions, tant esthétiques que philosophique, sans compter l’effet de réception que l’oeuvre procure sur celui qui la reçoit. En outre, la dimension spectaculaire est sous-entendue dans ce que Masson ambitionne d’être \enquote{une fête pour les yeux.}\footcite{memoiremonde}. Elle fait d’autre part allusion au procédé de jet de la peinture, le trait et la couleur comme jaillissement. 

	Cependant, André Masson précise une autre affinité très puissante entre Aragon et lui, déjà au moment de leur rencontre, et qu’il sous-entendrait même partager avec lui de manière plus profonde qu’avec les autres compagnons surréalistes : La peinture. 
	Familier de la peinture la plus hardie de notre temps er de celle qui l’a précédée, il ne faut pas oublier notre intérêt pour Géricault - peintre épique - et sa mise en lumière de Girodet, peintre étonnant et chassé de l’horizon pictural par les surréalistes depuis les premières années de l’impressionnisme. Redécouverte  courageuse et que j’aimerais approfondir si j’en avais les moyens critiques.
	
	Ce qui ressort de ce témoignage, c’est avant tout la valeur d’engagement dans le champ de l’art dans la défense d’un artiste particulier, qui n’est pas sans rappeler celle de l’engagement politique. Cette prise de risques est saluée par Masson à propos de la défense d’Aragon de Girodet, mais il est intéressant de souligner les limites qu’André Masson se reconnaît à propos d’un domaine qu’en tant que peintre il paraitrait évident de concevoir comme étant sa spécialité. Mais, face au sujet de la peinture, Masson conçoit non seulement Aragon en tant que théoricien de l’art, celui qui serait selon Masson le plus à même de la traiter. Les atomes crochus pour les figures de peintres à relégitimer confirme ce jeu de miroir entre l’artiste et l’écrivain, y compris écrivain d’art. Néanmoins, le jeu de miroirs ne signifie pas pour autant l’opacité totale, et, à propos des propres oeuvres de Masson, c’est l’énigme qui fascine tant Aragon :

\begin{verse}
Qu’entreprends-tu dis-moi le secret de tes images
J’essaye de surprendre en toi le murmure des mots
Mystérieux - N’ayant que mes yeux pour voir	
\footcite[p685]{ecritssurla}\end{verse}

	Le vertige d’Aragon devant une oeuvre de Masson tiendrait à l’envie pressente de se voir dévoiler l’énigme de l’oeuvre, tout en savourant celle-ci ne lui soit pas immédiatement manifestée. C’est le vertige des oeuvres de l’un qui produit sur l’autre, à la fois à travers les années et contre le temps lui-même, qui produit ce jeu de miroir qui explique cet aveu d’Aragon dans sa cantate : \enquote{André Masson / à qui tout ce que j’écris s’adresse.}\footcite[p692]{ecritssurla} Une telle révélation confirme d’autant plus le miroir dans lequel se retrouve un Aragon âgé vis-à-vis d’un ami qui connait la même longévité que les allusions à Elsa sont tout aussi présentes que ses allusions à sa mère et des fragments d’enfance. En somme, lorsque Aragon parle d’André Masson, à l’adresse d’André Masson, il en revient sur le mode d’un cycle à établir son propre portrait. Il en va de même pour l’écho de leur recherche esthétique opéré entre le vertige chez Aragon et la fête chez Masson. L’extase, l’intensité au point ultime du sentiment, suspendu dans le temps. 



