\chapter{Le lyrisme révolutionnaire, souffle insurrectionnel des luttes politiques dans \emph{Les Lettres françaises} :}

\section{Le croisement lyrisme révolutionnaire dans le réalisme socialiste  et la place d’André Masson dans \emph{Les Lettres françaises}:}

%Source importante : Les Lettres françaises [n°768- du 9 au 15 avril 1959] "Savoir aimer" par Aragon
\subsection{Article \emph{Savoir aimer}par Aragon}

Dans un troisième temps, à la lumière de la corrélation du travail esthétique et du message politique dans \emph{Les Lettres françaises}, il convient de mesurer le rôle de cet aspect du lyrisme révolutionnaire dans le traitement du réalisme socialiste. Cette notion dont Aragon se fait le grand représentant avec ses oeuvres romanesques du \emph{Monde réel} en France vient de ses voyages en URSS, et vise à ce que l’oeuvre littérature et celle des arts en général se fasse le miroir de la société du point de vue du prolétariat. Revisiter cette notion dans le journal des \emph{Lettres françaises} est particulièrement révélateur du point de vue de l’évolution d’Aragon dans le temps avec cette grande idée lorsqu’il doit peu à peu y renoncer surtout après 1956 où avec les révélations du Rapport de Khrouchtchev l’image de Staline est désacralisée. Cependant, le réalisme socialiste, lui, met du temps à s’éteindre complètement, notamment dans les articles de critique d’Aragon. 

	 Aragon publie son article \emph{Savoir aimer} un an après une autre grande enquête, celle déjà évoquée de 1958 : \emph{Qu’est-ce que l’avant-garde en 1958 ?}. Les influences du sujet sur cet article de 59 sont manifestes : Aragon semble répondre indirectement au sujet lorsqu’il rappelle à partir de son expérience dada : \enquote{c’était une mode du Mouvement Dada que d’affirmer avec Francis Picabia d’une certaine façon désabusée : \emph{Au bout du compte tout se classe}.}

%Qu’est-ce que l’avant-garde en 1958 ? Source : Les Lettres françaises [n°716- 3-9 avril 1958] / mettre l'illustration de la page de couverture. 

 Mais une autre question émerge dans l’argumentation d’Aragon pour évoquer ces oeuvres hors-normes devenues académiques, celle de l’identité :  
 \begin{quote}
  Mais je ne sais pourquoi, nous avons toujours le sentiment que ce sont les autres qui se sont trompés, et que nous jugeons mieux que nos prédécesseurs. Il n’y a en ce sens aucune garantie, et par exemple, dans une seule vie, on peut fort bien se trouver en contradiction avec soi-même.    
 \end{quote}

	Est-ce qu’on peut voir dans cette reconnaissance d’oeuvres contradictoires chez un même individu une possible mise à distance avec le réalisme socialiste ? Toujours est-il qu’à la fois en partant de ses convictions au temps de sa jeunesse et cette conception d’une identité altérée, Aragon semble revenir pourtant à son passé dada plutôt qu’il ne s’en éloigne. Le lien avec les oeuvres d’avant-gardes pour rentrer dans ce « classement » sont donc liés pour Aragon pas seulement aux tournants des mouvements artistiques, mais lié aussi aux mouvements d’identités : \enquote{Et tout, par exemple, ne se classe pas nécessairement en littérature. Il se déclasse, il se démonétise beaucoup.}

Ce procédé du classement, et ce processus de changements et de revirements du même classement des oeuvres, ce serait en somme une exploration sur soi. Et, même vis-à-vis de ces variations qui constituent un individu à travers les années. Aragon suggère même un vertige de l’identité, vertige comme finalité de la critique : 

\begin{quote}
 On voit au bout de quelques années se constituer de déroutantes échelles de valeurs, et soi-même on se met à douter de ce qu’on a pensé dans sa jeunesse. La durée d’une vie humaine est suffisante pour déconcerter l’esprit critique, et trop brève pour permettre au jugement de retrouver son équilibre.   
\end{quote}
 

	 On retrouve cette dimension fondamentale du temps, où les effets du temps produisent la déstabilisation de ces changements de perspectives depuis la jeunesse. L’entre-deux du temps ne peut aboutir qu’au vertige de la critique, et non au lieu commun de la critique plutôt dans la distanciation, le juste recul. Il peut paraitre paradoxal que l’article s’achève en éloge au réalisme socialiste, alors qu’à cette période Aragon commence à se tourner vers une autre conception du réalisme, qui n’est pas sans rappeler celui de Masson lorsqu’il peint ou dessine les hommes. Le verbe \emph{aimer} ,porté par le titre \emph{avoir aimer} et plusieurs fois marqué par un italique de soulignement, se substitue par cette distinction typographique à « critiquer ». Savoir aimer, c’est savoir véritablement critiquer :  

     \begin{quote}
       Je pense, pour ma part, que le goût est une chose essentiellement positive. Que la critique devrait, en matière de littérature, être une sorte de pédagogie de l’enthousiasme. Qu’un vrai critique est celui qui apprend à \emph{aimer}, et attention ! j’emploie toujours verbe aimer au sens fort, j’entends ici que les critiques ne font pas leur métier, parce qu’on ne les voit jamais les yeux cernés pour avoir lu un livre, même quand ils en disent du bien.    
     \end{quote}


	 Cette conception de la critique, très lyrique avec pour essence l’idée d’ \enquote{aimer},  est donc très proche de celle de Masson sur Baudelaire comme critique dans l’article de 1968 neuf ans plus tard. Aragon et Masson partagent cette idée fondamentale d’une subjectivité du critique qui perdrait toute distance avec le sujet pour au contraire se plonger dans le vertige que procure l’oeuvre. Comme chez Masson, pas seulement dans ses conceptions mais aussi dans son art, Aragon associe à cette exigence mentale une conséquence physique, due à l’obsession du critique sur le sujet, qui se répercuterait sur le système nerveux. Cet entrelacement de la pensée et du physique emmène logiquement la métaphore de l’acte sexuel, : \enquote{Et personne ne songe à vous traiter d’impuissants, parce qu’on ne vous entend pas crier, quand vous lisez les romans ou les nouvelles de ce temps-ci. Il m’arrive de penser que c’est pour le moins étrange.} La métahpore est intéressante si l'on se rappelle que le l'érotime valait comme symbole de liberté totale dans \emph{Le Con d'Irène}. 

     	Une révolution intérieure, comme retournement des sens, doit envahir le critique, avec cette image insaisissable du cri. La même image du cri que Masson figure pour représenter la révolte paysanne. Comme Masson le revendiquera dans quelques années, Aragon prône dans la passion que suscite la critique la prise de risques du jugement : \enquote{Bien parler d’un livre c’est peu. Il faut encore le situer. Oser dire, ceci restera}. 

 Or, ce choix de détermination ou pas sur la destinée d’une oeuvre peut déjà se concevoir comme un choix politique. Celui d’anticiper selon le jugement de valeur une vision à long terme sur l’oeuvre. Si l’oeuvre parlera aux générations futures. Le sens politique se confirme avec cette « envie partisane » selon l’expression d’Aragon qui implique le geste du choix, et et particulier ce rôle crucial du critique, celui qui peut influencer pour que la destinée d’une oeuvre se prolonge dans le temps : 

\emph{ce mécanisme indémontable de l’art, par quoi se fonde la grandeur de l’oeuvre, et son droit à ne pas mourir.} Le lyrisme de l’argumentation n’est pas seulement théorique, il est aussi typographique : Avec l’italique déjà abordé du verbe \enquote{aimer} qui insiste ainsi sur la force donnée au mot, mais aussi avec les majuscules, où « \enquote{J’AIME les choses bien faites} est redoublé quelques lignes plus tard par \enquote{JE ne sais pas comment vous avez la tête faite, et de quoi elle est peuplée : pour moi, j’ai toute la vie porté en moi des images dont rien ne pourrait me séparer.}Le lyrisme est clairement manifesté avec ces majuscules autour du \enquote{je} et du verbe sur l’expression des sentiments par excellence. Sans compter que le sens de ces deux phrases éloignées dans l’article tournent autour de la même notion, même si la seconde phrase est plus explicite : le lyrisme vient des images qui hantent le critique, d’une façon inconsciente qui peut évoquer la période surréaliste comme celle de dada au début de l’article. 

	Et pourtant, c’est encore par une sauvegarde du réalisme socialiste qu’Aragon ponctue finalement, comme le lieu même de l’amour : 
\begin{quote}
  l’amour pour que je le ressente, que je le partage doit  être réel, et réaliste l’art qui le décrit, et que c’est l’étrange calomnie que de prétendre que si ce réalisme a le socialisme pour soleil l’amour s’y doit étioler, quand c’est au contraire cet \emph{idéal qui donne à l’histoire réelle, la force même de l’amour}.   
\end{quote}


Il peut paraître paradoxal que pour aboutir au réalisme socialiste, Aragon soit revenu sur les autres formes de réalisme du temps de sa jeunesse. D’autant plus qu’en 1959, le réalisme socialiste va peu à peu dans ses oeuvres laisser la place à une autre perspective du réalisme. 

	Cet article qui manifeste l’entrelacement du lyrisme à une dimension politique sur deux aspects : une notion subjective propre au choix, celui de parti-pris de chercher à faire perdurer une oeuvre plutôt qu’une autre, ce qui se rapproche d’une politique éditoriale sur les choix d’articles et de mise-en-page d’un numéro. En particulier pour le journal \emph{Les Lettres françaises}, qui, par sa position avant tout culturelle, est nourrie en grande partie de critiques. C’est donc toute la ligne éditoriale du journal qui est explicité dans l’article, son \enquote{envie partisane}et la recherche du vertige dans les agencements d’articles comme dans le choix des sujets. D’autre part, on peut concevoir cet article comme un entre-deux, ou plutôt un mouvement pris dans la réflexion d’Aragon à propos du réalisme socialiste :  A la fois l’idée même du socialisme  pour Aragon qui procure le lyrisme et cette force subversive. Et, en même temps, des traces d’affinités pour une autre conception du réalisme font déjà pressentir les nouvelles réflexions romanesques en cours pour Aragon en tant qu’auteur. 

\subsection{Les échanges d'André Masson avec \emph{Les Lettres françaises }dans la transition avec la postion du réalisme socialiste du journal}

L’année 1959 semble être l’année de l’entre-deux, pour Aragon d’une part, mais pour Masson aussi, si l’on s’en tient à ses oeuvres exposées au Salon de Mai en mai 1959 : \enquote{un André Masson plein de fougue qui illustre la transition entre le surréalisme et l’abstraction}, d’après Georges Boudaille. Un même mouvement de transition s’opère chez Aragon comme chez Masson, entre les élans de retours vers des affinités premières, et les prémices d’une réflexion autre qui émerge de cet entre-deux. Même si Boudaille qualifie de \enquote{fougue} la force des oeuvres expressives de Masson, la toile représentée sur la page dans l’article, Un couple dans la nuit, revient dans un autre numéro pour illustrer un texte de Claude Durand. Toujours sur ce thème de l’entre-deux, entre le jour et la nuit, où le lyrisme du personnage qui erre dans les rues la nuit et revit des discussions de couples fait agir \emph{tUn couple dans la nuit} plus comme un écho à la réflexion du narrateur qu’à une illustration : 

%Salon de mai de 1959 Les Lettres françaises  [du 14 au 20 mai 1959]- Quoi de neuf au Salon de Mai ? par G. Boudaille


\begin{quote}
Ivresse des limites perdues. Mais, parce que je suis seul, aucune épouvante ne me saisit (c’est ce qui nous ressemble qui nous terrifie). L’angoisse ? Ce vide m’était encore tout à l’heure mon droit à la joie, presque une plénitude…A quoi est-ce que je crois ? 	
\end{quote}


