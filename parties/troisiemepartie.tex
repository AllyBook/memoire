\chapter{Le lyrisme révolutionnaire, souffle insurrectionnel des luttes politiques dans \emph{Les Lettres françaises} :}

\section{Le croisement lyrisme révolutionnaire dans le réalisme socialiste  et la place d’André Masson dans \emph{Les Lettres françaises}:}


Dans un troisième temps, à la lumière de la corrélation du travail esthétique et du message politique dans \emph{Les Lettres françaises}, il convient de mesurer le rôle de cet aspect du lyrisme révolutionnaire dans le traitement du réalisme socialiste. Cette notion dont Aragon se fait le grand représentant avec ses oeuvres romanesques du \emph{Monde réel} en France vient de ses voyages en URSS, et vise à ce que l’oeuvre littérature et celle des arts en général se fasse le miroir de la société du point de vue du prolétariat. Revisiter cette notion dans le journal des \emph{Lettres françaises} est particulièrement révélateur du point de vue de l’évolution d’Aragon dans le temps avec cette grande idée lorsqu’il doit peu à peu y renoncer surtout après 1956 où avec les révélations du Rapport de Khrouchtchev l’image de Staline est désacralisée. Cependant, le réalisme socialiste, lui, met du temps à s’éteindre complètement, notamment dans les articles de critique d’Aragon. 

	 Aragon publie son article \emph{Savoir aimer} un an après une autre grande enquête, celle déjà évoquée de 1958 : \emph{Qu’est-ce que l’avant-garde en 1958 ?}. Les influences du sujet sur cet article de 59 sont manifestes : Aragon semble répondre indirectement au sujet lorsqu’il rappelle à partir de son expérience dada : \enquote{c’était une mode du Mouvement Dada que d’affirmer avec Francis Picabia d’une certaine façon désabusée : \emph{Au bout du compte tout se classe}.}

%Qu’est-ce que l’avant-garde en 1958 ? Source : Les Lettres françaises [n°716- 3-9 avril 1958] / mettre l'illustration de la page de couverture. 