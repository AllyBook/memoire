\chapter{Le lyrisme révolutionnaire, souffle insurrectionnel des luttes politiques dans \emph{Les Lettres françaises} :}

\section{Le croisement lyrisme révolutionnaire dans le réalisme socialiste  et la place d’André Masson dans \emph{Les Lettres françaises}:}

\subsection{Article \emph{Savoir aimer}par Aragon}

Dans un troisième temps, à la lumière de la corrélation du travail esthétique et du message politique dans \emph{Les Lettres françaises}, il convient de mesurer le rôle de cet aspect du lyrisme révolutionnaire dans le traitement du réalisme socialiste. Cette notion dont Aragon se fait le grand représentant avec ses oeuvres romanesques du \emph{Monde réel} en France vient de ses voyages en URSS, et vise à ce que l’oeuvre littérature et celle des arts en général se fasse le miroir de la société du point de vue du prolétariat. Revisiter cette notion dans le journal des \emph{Lettres françaises} est particulièrement révélateur du point de vue de l’évolution d’Aragon dans le temps avec cette grande idée lorsqu’il doit peu à peu y renoncer surtout après 1956 où avec les révélations du Rapport de Khrouchtchev l’image de Staline est désacralisée. Cependant, le réalisme socialiste, lui, met du temps à s’éteindre complètement, notamment dans les articles de critique d’Aragon. 

	 Aragon publie son article \emph{Savoir aimer}\footcite{savoiraimer} un an après une autre grande enquête, celle déjà évoquée de 1958 : \emph{Qu’est-ce que l’avant-garde en 1958 ?}\footcite{avantgarde}. Les influences du sujet sur cet article de 59 sont manifestes : Aragon semble répondre indirectement au sujet lorsqu’il rappelle à partir de son expérience dada : \enquote{c’était une mode du Mouvement Dada que d’affirmer avec Francis Picabia d’une certaine façon désabusée : \emph{Au bout du compte tout se classe}.}

 / mettre l'illustration de la page de couverture. 

 Mais une autre question émerge dans l’argumentation d’Aragon pour évoquer ces oeuvres hors-normes devenues académiques, celle de l’identité :  
 \begin{quote}
  Mais je ne sais pourquoi, nous avons toujours le sentiment que ce sont les autres qui se sont trompés, et que nous jugeons mieux que nos prédécesseurs. Il n’y a en ce sens aucune garantie, et par exemple, dans une seule vie, on peut fort bien se trouver en contradiction avec soi-même.    
 \end{quote}

	Est-ce qu’on peut voir dans cette reconnaissance d’oeuvres contradictoires chez un même individu une possible mise à distance avec le réalisme socialiste ? Toujours est-il qu’à la fois en partant de ses convictions au temps de sa jeunesse et cette conception d’une identité altérée, Aragon semble revenir pourtant à son passé dada plutôt qu’il ne s’en éloigne. Le lien avec les oeuvres d’avant-gardes pour rentrer dans ce « classement » sont donc liés pour Aragon pas seulement aux tournants des mouvements artistiques, mais lié aussi aux mouvements d’identités : \enquote{Et tout, par exemple, ne se classe pas nécessairement en littérature. Il se déclasse, il se démonétise beaucoup.}

Ce procédé du classement, et ce processus de changements et de revirements du même classement des oeuvres, ce serait en somme une exploration sur soi. Et, même vis-à-vis de ces variations qui constituent un individu à travers les années. Aragon suggère même un vertige de l’identité, vertige comme finalité de la critique : 

\begin{quote}
 On voit au bout de quelques années se constituer de déroutantes échelles de valeurs, et soi-même on se met à douter de ce qu’on a pensé dans sa jeunesse. La durée d’une vie humaine est suffisante pour déconcerter l’esprit critique, et trop brève pour permettre au jugement de retrouver son équilibre.   
\end{quote}
 

	 On retrouve cette dimension fondamentale du temps, où les effets du temps produisent la déstabilisation de ces changements de perspectives depuis la jeunesse. L’entre-deux du temps ne peut aboutir qu’au vertige de la critique, et non au lieu commun de la critique plutôt dans la distanciation, le juste recul. Il peut paraitre paradoxal que l’article s’achève en éloge au réalisme socialiste, alors qu’à cette période Aragon commence à se tourner vers une autre conception du réalisme, qui n’est pas sans rappeler celui de Masson lorsqu’il peint ou dessine les hommes. Le verbe \emph{aimer} ,porté par le titre \emph{avoir aimer} et plusieurs fois marqué par un italique de soulignement, se substitue par cette distinction typographique à « critiquer ». Savoir aimer, c’est savoir véritablement critiquer :  

     \begin{quote}
       Je pense, pour ma part, que le goût est une chose essentiellement positive. Que la critique devrait, en matière de littérature, être une sorte de pédagogie de l’enthousiasme. Qu’un vrai critique est celui qui apprend à \emph{aimer}, et attention ! j’emploie toujours verbe aimer au sens fort, j’entends ici que les critiques ne font pas leur métier, parce qu’on ne les voit jamais les yeux cernés pour avoir lu un livre, même quand ils en disent du bien.    
     \end{quote}


	 Cette conception de la critique, très lyrique avec pour essence l’idée d’ \enquote{aimer},  est donc très proche de celle de Masson sur Baudelaire comme critique dans l’article de 1968 neuf ans plus tard. Aragon et Masson partagent cette idée fondamentale d’une subjectivité du critique qui perdrait toute distance avec le sujet pour au contraire se plonger dans le vertige que procure l’oeuvre. Comme chez Masson, pas seulement dans ses conceptions mais aussi dans son art, Aragon associe à cette exigence mentale une conséquence physique, due à l’obsession du critique sur le sujet, qui se répercuterait sur le système nerveux. Cet entrelacement de la pensée et du physique emmène logiquement la métaphore de l’acte sexuel, : \enquote{Et personne ne songe à vous traiter d’impuissants, parce qu’on ne vous entend pas crier, quand vous lisez les romans ou les nouvelles de ce temps-ci. Il m’arrive de penser que c’est pour le moins étrange.} La métahpore est intéressante si l'on se rappelle que le l'érotime valait comme symbole de liberté totale dans \emph{Le Con d'Irène}. 

     	Une révolution intérieure, comme retournement des sens, doit envahir le critique, avec cette image insaisissable du cri. La même image du cri que Masson figure pour représenter la révolte paysanne. Comme Masson le revendiquera dans quelques années, Aragon prône dans la passion que suscite la critique la prise de risques du jugement : \enquote{Bien parler d’un livre c’est peu. Il faut encore le situer. Oser dire, ceci restera}. 

 Or, ce choix de détermination ou pas sur la destinée d’une oeuvre peut déjà se concevoir comme un choix politique. Celui d’anticiper selon le jugement de valeur une vision à long terme sur l’oeuvre. Si l’oeuvre parlera aux générations futures. Le sens politique se confirme avec cette « envie partisane » selon l’expression d’Aragon qui implique le geste du choix, et et particulier ce rôle crucial du critique, celui qui peut influencer pour que la destinée d’une oeuvre se prolonge dans le temps : 

\emph{ce mécanisme indémontable de l’art, par quoi se fonde la grandeur de l’oeuvre, et son droit à ne pas mourir.} Le lyrisme de l’argumentation n’est pas seulement théorique, il est aussi typographique : Avec l’italique déjà abordé du verbe \enquote{aimer} qui insiste ainsi sur la force donnée au mot, mais aussi avec les majuscules, où « \enquote{J’AIME les choses bien faites} est redoublé quelques lignes plus tard par \enquote{JE ne sais pas comment vous avez la tête faite, et de quoi elle est peuplée : pour moi, j’ai toute la vie porté en moi des images dont rien ne pourrait me séparer.}Le lyrisme est clairement manifesté avec ces majuscules autour du \enquote{je} et du verbe sur l’expression des sentiments par excellence. Sans compter que le sens de ces deux phrases éloignées dans l’article tournent autour de la même notion, même si la seconde phrase est plus explicite : le lyrisme vient des images qui hantent le critique, d’une façon inconsciente qui peut évoquer la période surréaliste comme celle de dada au début de l’article. 

	Et pourtant, c’est encore par une sauvegarde du réalisme socialiste qu’Aragon ponctue finalement, comme le lieu même de l’amour : 
\begin{quote}
  l’amour pour que je le ressente, que je le partage doit  être réel, et réaliste l’art qui le décrit, et que c’est l’étrange calomnie que de prétendre que si ce réalisme a le socialisme pour soleil l’amour s’y doit étioler, quand c’est au contraire cet \emph{idéal qui donne à l’histoire réelle, la force même de l’amour}.   
\end{quote}


Il peut paraître paradoxal que pour aboutir au réalisme socialiste, Aragon soit revenu sur les autres formes de réalisme du temps de sa jeunesse. D’autant plus qu’en 1959, le réalisme socialiste va peu à peu dans ses oeuvres laisser la place à une autre perspective du réalisme. 

	Cet article qui manifeste l’entrelacement du lyrisme à une dimension politique sur deux aspects : une notion subjective propre au choix, celui de parti-pris de chercher à faire perdurer une oeuvre plutôt qu’une autre, ce qui se rapproche d’une politique éditoriale sur les choix d’articles et de mise-en-page d’un numéro. En particulier pour le journal \emph{Les Lettres françaises}, qui, par sa position avant tout culturelle, est nourrie en grande partie de critiques. C’est donc toute la ligne éditoriale du journal qui est explicité dans l’article, son \enquote{envie partisane}et la recherche du vertige dans les agencements d’articles comme dans le choix des sujets. D’autre part, on peut concevoir cet article comme un entre-deux, ou plutôt un mouvement pris dans la réflexion d’Aragon à propos du réalisme socialiste :  A la fois l’idée même du socialisme  pour Aragon qui procure le lyrisme et cette force subversive. Et, en même temps, des traces d’affinités pour une autre conception du réalisme font déjà pressentir les nouvelles réflexions romanesques en cours pour Aragon en tant qu’auteur. 

\subsection{Les échanges d'André Masson avec \emph{Les Lettres françaises }dans la transition avec la postion du réalisme socialiste du journal}

L’année 1959 semble être l’année de l’entre-deux, pour Aragon d’une part, mais pour Masson aussi, si l’on s’en tient à ses oeuvres exposées au Salon de Mai\footcite{salondemai} en mai 1959 : \enquote{un André Masson plein de fougue qui illustre la transition entre le surréalisme et l’abstraction}, d’après Georges Boudaille. Un même mouvement de transition s’opère chez Aragon comme chez Masson, entre les élans de retours vers des affinités premières, et les prémices d’une réflexion autre qui émerge de cet entre-deux. Même si Boudaille qualifie de \enquote{fougue} la force des oeuvres expressives de Masson, la toile représentée sur la page dans l’article, Un couple dans la nuit, revient dans un autre numéro pour illustrer un texte de Claude Durand. Toujours sur ce thème de l’entre-deux, entre le jour et la nuit, où le lyrisme du personnage qui erre dans les rues la nuit et revit des discussions de couples fait agir \emph{tUn couple dans la nuit} plus comme un écho à la réflexion du narrateur qu’à une illustration : 


\begin{quote}
Ivresse des limites perdues. Mais, parce que je suis seul, aucune épouvante ne me saisit (c’est ce qui nous ressemble qui nous terrifie). L’angoisse ? Ce vide m’était encore tout à l’heure mon droit à la joie, presque une plénitude…A quoi est-ce que je crois ? 	
\end{quote}


 Or, l’angoisse, plus qu’un sentiment, est un véritable moyen d’expression dans l’art d’André Masson. Dans l’article de Georges Boudaille comme dans le texte de Claude Durand, le titre de l’oeuvre de Masson apparemment serein est ainsi qualifié par l’énergie et l’errance, thème romantique. Il est vrai que, si les silhouettes de l’homme et de la femme sont immédiatement perceptibles, c’est par leurs traits appuyés plutôt que par la représentation d’une chair. Leur main entrelacée, par un croisement de lignes, n’en figure qu’une seule. Paradoxalement, les lignes apportent l’érotisme à leur corps nu, de façon plus directe que s’il s’était agit d’une représentation réaliste et fidèle d’un homme et d’une femme. La force symbolique que procure les lignes pour former les corps nus rapprochent ce couple plus d’une Idée au sens symbolique, plutôt que de personnages, chargés de jouer un rôle quelconque dans l’oeuvre.  


Dans un numéro de 1952, Pierre Descargues qualifie l’esthétique d’André Masson de \enquote{réel fantastique} dans son article \emph{André Masson et le réel fantastique}. 1952 est une année d’entre-deux pour l’aspect éditorial du journal : Aragon dirige le journal en 1953, mais il prend la direction en 1951 de la rubrique \emph{Tous les arts}. Pierre Descargues y occupe conjointement avec Aragon une chronique, \emph{A travers les galeries}. Ainsi, tout comme d’autres chroniqueurs tels que Georges Boudaille ou Georges Besson, Descargues connaît intimement Aragon et André Masson. 
% Les Lettres françaises  [n°411- du 24 avril au 1er mai 1952] André Masson et le réel fantastique  par Pierre Descarges

	Il s'avère que, au début des années 50, la politique éditoriale est toujours tournée vers une politique de paix dans un contexte de guerre froide, avec le symbole des colombes massivement représenté depuis 1948, et la défense du réalisme socialiste. Est-il possible que cette page de la rubrique des Arts propose implicitement un croisement entre le réalisme socialiste et le réel fantastique ? En ouvrant son article avec des mots de Masson à propos de l’art de figurer le paysage, c’est le lien étroit entre le réel et l’imaginaire qui est pointé : 

 \begin{quote}
Prenez avec lui quelques libertés. Ne détournez pas les yeux quand le soleil descend dans l’arbre et fait basculer l’horizon. Même imaginaire, ne le débitez pas en tranches géographiques, ne le bornez pas à gauche, à droite. En haut, faites éclater l’immuable plafond .	
\end{quote}

	 En somme, Masson en appelle au contraire de l’ordre, de la catégorisation. Non seulement il préconise à l’artiste des « libertés », mais la Nature elle-même comporte une part subversive pour lui. Ce qui explique le geste de l’artiste subversif aussi pour Masson qui ne recherche pas la mimétique de la Nature, un refus d’un ordre classique et aligné de la composition, pour au contraire \enquote{éclater l’immuable plafond}.

	  Cette idée de jaillissement rappelle même plus l’abstraction que la figuration, à ceci près qu’une fois encore il s’agit de se détourner du mimétique pour l’expression de la Nature, son éventuel véritable réalisme. Or, ce réalisme ne peut être possible qu’avec ce facteur fantastique, qui dans l’article est plutôt tourné vers l’idée de l’imaginaire et de l’onirique. On relève donc la dimension paradoxale d’un réalisme qui advient par l’imaginaire et le rêve dans le but de figurer l’essence de la Nature, ce qu’elle veut signifier, plutôt que sa copie. C’est qui fait écrire au critique que Masson est en somme un insaisissable : 
 
L’allusion à Turner est d’autant plus parlante pour les quelques lignes que Masson lui avait sacré au début de cette même année 1952 (clin d’oeil de Descargues ? ) : \enquote{Disparition de la pesanteur}. Ces propos sur Turner sont essentiels pour comprendre qu’en refusant le mimétique, Masson a aussi délibérément choisit de de composer avec des lignes plutôt qu’avec l’effet de masse. La forme faite de lignes fait deviner plus qu’il ne le signifie directement, puisque le sujet est dépossédé de sa masse corporelle. Sans compter qu’on peut considérer Turner comme faisant de l’impressionniste avant l’heure, ce qui fait de son mouvement aux couleurs chaudes un art insaisissable à son époque d’une façon similaire à André Masson. 

	Mais il s’agit de voir comment le \enquote{réel fantastique} de Masson s’inscrit dans le projet du réalisme socialiste dont son grand représentant est le directeur, donc l’organisateur, de cette page de Tous les Arts. Or, la part de romantisme révolutionnaire est étroitement liée au réalisme socialiste. Où, si l’on prend la question à l’inverse, y a-t-il du socialisme dans le réel fantastique ? 

	Reynald Lahanque offre une première piste sur les motivations éditoriales du journal, et plus précisément d’Aragon comme responsable de cette rubrique Tous les Arts, quelques mois avant d’être officiellement directeur de tout l’hebdomadaire, dans sa thèse sur le réalisme socialiste en France : 

\begin{quote}
Le terme même de \enquote{réalisme socialiste}se rencontre assez rarement dans \emph{Les Lettres françaises}, mais la problématique que le terme recouvre y est très largement présente. Dirigé dans les faits par Aragon et Daix, l'hebdomadaire culturel fait toute sa place, sur un plan plus général, aux thèses et aux thèmes de la propagande communiste de la guerre froide. Il garde en même temps l'ambition de toucher un public plus large que celui des militants, et conserve des collaborateurs moins engagés politiquement et capables d'élargir la gamme de ses centres d’intérêt. \footcite{}\end{quote}

	Comme pour toute politique, le choix éditorial met en valeur un élément au détriment d’un autre. A dessein d’élargir les lecteurs à un public qui ne viendrait pas nécessairement de l’appareil du PCF, le terme « réalisme socialiste » subit un profond paradoxe : Il est plus que jamais à l’ordre du jour dans ce contexte de guerre froide, tout en restant discret en raison de sa connotation immédiatement politique. On retrouve ici ce qu’Aragon revendique quelques années plus tard en 59 comme l’\enquote{envie partisane} comme essence du travail de critique. 

Or, si l’\enquote{envie partisane} existe déjà en 1952, elle ne doit pas paraître invasive, puisqu’elle définit une idéologie précise en contradiction avec la politique éditoriale d’ouverture à un plus large public. Face à cette contradiction entre les attentes politiques et les attentes éditoriales, le mot d’ordre qui vient à la place envahir le journal, mais qui sous-tendrait au réalisme socialiste, c’est le mot \enquote{Paix}, omniprésent depuis les années 48, et d’une portée large, voire unanime. A peine quelques années après la Seconde Guerre Mondiale, qui voudrait encore la guerre ? Mais, sur le plan artistique, le \enquote{réel fantastique} pour désigner l’art d’André Masson pourrait lui-même aboutir idéologiquement à une forme dérivée du réalisme socialiste. Avec, cependant, une qualification onirique à ce réalisme qui n’apparente pas l’expression directement au une théorie politique. 

Jusqu’à son terme, la dimension proprement lyrique de l’article de Descargues, avec cette métaphore du vent pour incarner à la fois les représentations et la méthode de Masson, semble appuyer la rêverie et s’éloigner de tout aspect politique : \enquote{Il est allé chez lui dans le vent, dans les immenses mouvements de ciel et de la terre, dans tout ce qu’à soigneusement évité la peinture moderne jusqu’à aujourd’hui. Parce que sa personnalité est trop originale.} On est apparemment plus proche d’une philosophie telle les \emph{Rêveries d’un promeneur solitaire} de Rousseau que du réalisme socialiste. 

	Cependant, la liberté est aussi un concept politique fondamental, et c’est bien celle-ci qu’aspire à faire rêver au lecteur le mouvement lyrique. L’adverbe \enquote{trop} pour conclure sur la dimension atypique de Masson ne peut pas être anodine. En substance, l’art d’André Masson est un débordement. Peut-on alors supposer que le \enquote{réel fantastique} est un mode de représentation du réalisme socialiste ? La question peut trouver des réponses contradictoires, dans cette année 1952, où, dans un article d’éloge à la remise du Prix Staline au roman \emph{Le premier choc d’André Stil}, Aragon rappelle, non sans apparenter cette définition en partie à Staline, ce qu’est le réalisme socialiste : 
	\begin{quote}
	Le réalisme socialiste, étant la méthode de base de la littérature et de la critique soviétique, exige de l’artiste une représentation véridique, historiquement concrète de la réalité dans son développement révolutionnaire. De plus, le caractère véritable et historiquement concret de cette représentation artistique de la réalité doit se combiner avec le devoir de transformation idéologique et d’éducation des masses dans l’esprit du socialisme.\footcite{}\end{quote}

	%Source à mettre :   Les Lettres françaises [n°409- 10 avril 1952] Parenthèse sur les Prix Staline par Aragon 

	
	 Une telle définition demanderait à se poursuivre par la définition cette fois du socialisme. Est-il d’ailleurs le même en URSS qu’en France, si on prend en compte le fait que la politique du parti communiste diffère sur certaines caractéristiques d’un pays à l’autre ? Reynald Lahanque pointe d’ailleurs la pratique du réalisme socialiste d’Aragon dans les romands du \emph{Monde réel}, peu représentatifs des héros communistes-types du réalisme socialiste et de scénario d’apprentissage par le parti qui en serait attendu d’après la définition. Mais, vis-à-vis du \enquote{réel fantastique}, c’est la question de la représentation du réalisme socialiste, \enquote{véridique, historique, concrète de la réalité}.

 Cet appel à la figuration manifeste et plutôt naturaliste contraste donc avec l’aspect \enquote{fantastique} de Masson. Cependant, si l’on s’arrête sur les louanges d’Aragon sur \emph{Le premier choc}, l’aspect très pragmatique établi dans cette définition du réalisme socialiste est substitué à une autre forme de réalisme : \enquote{Je veux parler de sa façon de décrire les personnages. Ou plutôt de ne pas les décrire}. On commence par cette suggestion d’une autre forme \enquote{concrète} du réel qui ne saurait pas une forme naturaliste à se rapprocher par la conception d’André Masson sur la représentation des paysages. Et, ce constat sur lequel s’appuie Aragon repose sur fondement analogue au « réel fantastique » de l’artiste de Descarges, c’est-à-dire figurer non pas le sujet en tant que tel, mais ce qu’exprime le sujet : \enquote{Parce que justement, ici, la ressemblance ne tient  pas à un \emph{trait} qui se répète. Mais à la nature complexe de l’homme décrit.}Cette phrase sur l’oeuvre de Stil pourrait se confondre avec celles de Masson : Contre le sujet figé du mimétiques, et la recherche de l’essence du sujet, sa nature. 

	Un tel projet ne peut que reposer en partie sur l’imagination pour représenter l’intérieur du sujet. Ce qu’Aragon dans ce même article nomme le \enquote{ypage d’âme} : \enquote{Les personnages du Premier Choc sont socialement définis et individuellement distingués par ce que j’appellerai le \emph{typage d’âme}. C’est à leur façon de penser, c’est au contenu de leur pensée, socialement définie- et au caractère de chacun, que l’auteur a fait appel pour fixer leurs images.}. Cette analogie entre \emph{Le premier choc} de Stil et l’oeuvre d’André Masson illustre la conviction d’Aragon au début de l’article qu’avec le réalisme socialiste, on peut parler de la littérature pour évoquer l’art, et réciproquement.  Le choix stylistique de Stil est d’ailleurs bien plus qu’un détail pour Aragon, et peut-être est-ce même l’un des fondements qui permet à cette notion son \enquote{ développement révolutionnaire} comme finalité, d’après sa définition : \enquote{“Ce qui se passe dans leurs yeux est plus important que leur couleur“…Je vous dis que cette page que j’ai recopiée a valeur de manifeste.} La fonction poétique de la phrase n’est pas anodine, c’est même elle qui lui confère une forme de maxime. Le \enquote{typage d’âme} pourrait donc être le fil directeur à la fois du réalisme socialiste et des convictions d’André Masson. Dès lors, la notion de concret est complètement redessinée. Cependant, si une nature révolutionnaire associée à une forme lyrique est commune à la fois l’article d’Aragon sur Stil et le réel fantastique de Masson, on peut distinguer cette finalité révolutionnaire : Le réalisme socialiste comporte dans la suite de sa définition une valeur éducative, qui est peut-être l’élément que l’on ne retrouve pas de façon évidente dans le \emph{Monde réel}. André Masson réclame de faire \enquote{éclater l’immuable plafond}, mais pas à dessein nécessairement éducatif, tout au contraire. Dans l’exemple du paysage, celui-ci a la forme du fameux que l’on retrouve symbolisé dans beaucoup de ses oeuvres, et cette expression du mouvement est une finalité en soi. La rationalisation du mouvement ne semble ni attendue ni souhaitée. 


Or, la définition du réalisme socialiste dans cet article d’Aragon pourrait comporter le paradoxe de sa finalité : \enquote{Le développement révolutionnaire} peut-il se concilier avec la seconde partie de la définition, \enquote{se combiner avec le devoir de transformation idéologique et d’éducation des masses dans l’esprit du socialisme.} ? Le mouvement révolutionnaire comporte en substance une notion de liberté, rattachée également à André Masson dans l’article sur le réel fantastique et à son art en général, semble difficilement rentrer dans la finalité de contrôle et d’ordre que sous-entend \enquote{l’éducation des masses}. C’est pourquoi, tout en rappelant lui-même cette définition et l’influence de Staline, si l’on évoque les oeuvres du \emph{Monde réel}, on peut se demander si les oeuvres réalistes socialistes d’Aragon n’ont pas elles-mêmes tant recherchées \enquote{l’éducation des masses} que le \enquote{développement révolutionnaire} des personnages. 

\subsection{Croisement de l'article \emph{Le réalisme socialiste n'est pas mort} par Araron et l'orientation de Masson dans l'article \emph{A trvaers les galeries} par Pierre Descargues : 

	Croiser un article de critique littéraire et un autre sur les expositions dans les galeries d’art en 1957 montre la grande influence de la foi communiste du directeur du journal, Aragon, cette année-là. Les livres vont êtres analysés, comme l’indique très clairement Aragon dans son titre \emph{Le réalisme socialiste n’est pas mort}, son commentaire sur les livres \enquote{L’Homme ne vit pas seulement de pain}, de Vladimir Doudintsev, et de \emph{L’Or} de Boris Polevoï se fait dans l’éclairage du réalisme socialiste. Ce qui n’est pas anodin : Les oeuvres sont louées pour leur valeur réaliste socialiste. 

%SOurce *critique littéraire :  Les Lettres françaises [n°673- du 30 mai au 5 juin 1957]  Le réalisme socialiste n’est pas mort par Aragon. / *galeries d'art : Les Lettres françaises, [n°669- du 2 au 8 mai 1957]- A travers les galeries, par Pierre Descargues


	Or, comme dans beaucoup d’illustrations des \emph{Lettres françaises }lorsqu’il s’agit d’André Masson, c’est un dessin qui représente ses oeuvres dans un numéro quelques mois plus tôt dans la rubrique \emph{A travers les galeries}, par Pierre Descargues. Un dessin, inachevé qui plus est. Un rejoint là encore, sur le plan artistique cette fois, une conception de la figuration étroitement liée à la politique communiste, et à laquelle le réalisme socialiste peut en partie exercer une influence : celle de la préférence au geste du dessin plutôt que de la peinture à l’huile. D’autant plus du dessin inachevé, plus proche encore de la genèse de l’oeuvre et du geste originel de l’artiste. 

	Et pourtant, Masson peut-il être un représentant du réalisme socialiste ou de la politique du parti communiste, alors que sa figuration d’une réalité autre rend ambiguë la frontière entre la figuration et l’abstraction :
	
	\begin{quote}
	 Hors d’un trait nerveux, de rayonnements colorés, remarquables, vous ne reconnaîtrez pas Masson à son style. Ce peintre ne s’est pas laissé peindre à cette mode du graphisme en quoi l’on veut que chacun se résume. Il ne veut pas non plus être le peintre de quelques accords de couleurs. Si jamais artiste a fui le style, c’est bien lui. Aussi trouve-t-ton plus d’un tableau déconcertant dans cette exposition.	
	\end{quote}
	
	 Masson à contre-courant, donc. Le dessin au-dessus des lignes de la chronique n’est lui-même pas immédiatement identifiable, et c’est d’abord des figures abstraites qui apparaissent. Néanmoins, et c’est probablement en raison du choix de représenter une réalité autre, si le sujet n’est pas immédiatement identifié, il se devine. 

Il se devine par le biais de ce qui justement évoque dans un premier temps l’abstraction : Les lignes, le tracé. Le corps d’une jeune femme nue en pleine nature apparaît donc dans les courbes des lignes. Une telle scène présenterait donc un sujet classique, mythologique ou sur la femme, deux thèmes chers à Masson, mais totalement redécouverts avec un nouveau regard de la part de l’observateur qui a deviné le sujet à la fois caché et révélé par le trait. Or, non seulement la figuration existe, mais puise son efficacité dans l’ambiguïté, mais il en de même pour le traitement des sujets : 

\begin{quote}
il s’attache à exprimer ces idées simples qui sont aussi des sensations et des spectacles d’une complexité formidable. Tant de choses tiennent-elles dans la peinture ? Il les y met, en tout cas, et trop souvent nous nous trouvions que devant un graphisme nerveux, devant quelques virgules nageant dans la couleur, devant de puissants mouvements.  	
\end{quote}
 
	On remarque d’ailleurs que les deux aspects sont tout autant des compliments pour Pierre Descargues : D’une part, la \enquote{simplicité}, avec des sujets au titre très bref généralement sur un sentiment, un personnage de mythologie, les femmes, les scènes de massacres. 


	D’autre part, \enquote{des spectacles d’une complexité formidable}. Masson rechercherait-il avec une autre forme du réel une autre représentation du sublime ?  Or, si l’on regarde le dessin publié au-dessus de l’article, la simplicité apparente s’accompagne si l’on suite les traits jaillis du corps féminin une grande part de détails : La courbe de la tête de la femme penchée, les motifs autour d’elle et à ses pieds. Des personnages inachevés surgissent même dans les airs. Ils surgissent parce qu’on ne les distingue pas tout de suite, et l’on ne sait pas immédiatement s’ils sont produits par l’hallucination du spectateur ou bien réellement produits par fragments par le mouvement des traits. 

	Avec cette création de personnages, les lignes ne composent pas uniquement le sujet, mais le racontent, mais comme le ferait non pas un narrateur mais plusieurs, plus proches de la polyphonie due à des voix mélangées. Il peut ainsi paraitre paradoxal qu’à l’heure du réalisme socialiste, ce dessin, certes pleinement représentatif du geste du dessinateur en train de créer comme l’apprécie le communisme, se rapproche plutôt du dessin automatique conçu par les surréalistes. On peut se demander devant ce dessin d’André Masson si le sujet de cette jeune femme nue était réfléchi avant le geste du dessin, ou si c’est justement l’inconscient des traits qui ont formé le sujet. La référence de Descargues au traitement de la couleur est intéressante, puisque le choix du dessin justement privilégie la forme, les fameuses \enquote{quelques virgules nageant dans la couleur}. Les couleurs du dessin révèleraient-elles alors une toute autre oeuvre ? Or, les couleurs sont ancrées elle-mêmes dans le tourbillon de la forme, indissociables du mouvement. Ce qui justifie le choix des \emph{Lettres françaises}, fait pour beaucoup d’artistes, mais particulièrement pour André Masson, de privilégier ses dessins et esquisses aux tableaux : Le mouvement figure à la fois le tracé et la couleur. Il est la composition dans sa totalité. Pour autant, pourquoi André Masson n’incarne pas pour autant la ligne du réalisme socialiste ? 


On se rappelle qu’en 1952, Aragon précise dans son article sur le prix Staline décerné au \emph{Premier Choc} d’André Stil non seulement la définition du réalisme socialiste, mais aussi que celui peut évoquer la peinture en parlant d’un livre et réciproquement. Or, dans l’article \emph{Le réalisme socialiste n’est pas mort} de 1957, Aragon commente deux oeuvres littéraires avec une distinction sur leur vocation romanesque : \enquote{Cela dit, il y a deux sortes de romans : ceux qui se racontent, comme celui de Doudintsev (2); ces dont on ne peut dire que le thème, comme celui de Polevoï}. Cette distinction est intéressante au regard des oeuvres de Masson, dont le titre désignerait plutôt un thème. Pourtant, si on regarde la scène fantastique de la femme nue dans la nature du \emph{Dessin }publié pour la rubrique à Travers les galeries quelques semaines plus tôt, peut-on être sûr que les traits ne racontent rien ? 

	Les traits vifs racontent avant même le sujet quelque chose de l’état d’André Masson en train de dessiner. Si narration il y a, elle n’est ni réfléchie ni anticipée, pour privilégier au contraire la vitesse du geste de création. D’après Bernard Noël, à propos des dessins automatiques de Masson, la vitesse c’est à la fois ce qui fait de Masson une figure emblématique du dessin automatique, tout en le mettant à part des autres artistes surréalistes : \enquote{Cette vitesse-là- celle même de l’automatisme - fait bien d’André Masson le premier peintre surréaliste, et cependant elle travaille à l’éloigner du groupe parce que l’explosante en elle-même n’est jamais fixe.}Ainsi, même avec l’empreinte indélébile du surréalisme dans ce Dessin, André Masson est à la fois une figure d’autorité sans pour autant être un modèle, puisque son style n’est pas suivi même pas les auteurs de dessins automatiques de sa génération. 

	Mais, en 1957, présenter ce \emph{Dessin} révélateur du tracé originel est sans doute la seule caractéristique commune entre André Masson et un réalisme socialiste qui n’est plus en plein essor en 1952 mais plutôt au stade de survie, comme l’induit le titre \emph{Le réalisme socialiste n’est pas mort}. Le lyrisme est cependant de première importance aussi bien dans ce \emp{Dessin} de Masson comme dans ces critiques de livres réalistes socialistes par Aragon. Mais la différence notable de l’expression lyrique tient probablement au rapport du temps : André Masson exprime des émotions dans l’instantanéité de l’instant présent. Le sujet, cette jeune femme n’existe en tant que telle parce que dessinée dans la vitesse du temps présent. La mobilité de son corps en témoigne, et pas un élément même parmi les personnages inachevés ou apparement naturels n’est statique. Or, la critique littéraire des oeuvres russes réalistes socialistes par Aragon ne crée par le lyrisme par le mouvement vif du présent, mais par la nostalgie, le passé : 

\begin{quote}
…c’est la donnée initiale du livre, mais comment suivre le détaille leurs aventures, restituer le tragique de cette histoire, rendre le parfum de la terre russe, de ses forêts dans l’été de 1941, le paysage bouleversant et boulversé, les passages d’hommes et d’animaux sur les pistes secrètes à deux pas de l’armée d’invasion en marche ?  	
\end{quote}	


Le fait que cette mélancolie soit vouée à l’URSS n’est surement pas anodin, et l’on pourrait presque entendre dans ce romantisme des lieux russes la propre nostalgie d’Aragon pour le pays dans son poème \emph{Hourra l’Oural} de 1934 : \enquote{C’est troublant / c’est tout à fait tremblant / Simplement sous leurs pieds la terre /s’était mise à se souvenir / L’Oural rêvait}. Comme dans la critique du livre \emph{textL’Or de Polevoï}, le lyrisme se construit autour de la terre, le symbole d’un lieu natal mais qui est aussi un espace ancré, fixe, là où les traits de Masson ne cherchent pas à retrouver le passé sacralisé. Le visage du couple, d’une femme est chez Polevoï comme dans \emph{Hourra l’Oural} mêlé aux racines russes, ce qui renforce l’imaginaire romantique autour de la terre natale, ou symboliquement natale dans le cas d’Aragon. Le trait de Masson, lui, est en partie éphémère, avec ses personnages inachevés, sans origine distincte puisque le tracé est emmêlé. Ainsi, si le lyrisme est omniprésent dans le Dessin comme dans la critique d’oeuvres réalistes socialistes, son traitement est nuancé.

%Source : Hourra l’Oural, Le Capital Volant, Encore des paroles en l’air, 1934


	D’autre part, la critique d’Aragon repose sur la nécessité d’un héros collectif et réhabilite le genre noble par excellence, l’épique : 
	\begin{quote}
	et brusquement nous nous rendons compte que l’acte héroïque a été et ne pouvait être qu’un acte collectif, et que ce livre n’a pas été le roman de deux, trois, quatre ou cinq hommes et femmes particuliers, mais l’épopée d’un peuple, le roman d’un peuple, au sens du \emph{Roman de Thèbes}, la \emph{Chanson de Roland}.	
	\end{quote}
	
	
	 Or, si l’épique est d’une part un genre peu présent dans la production littéraire contemporaine, c’est aussi pour Aragon en tant que directeur des \emph{Lettres françaises} de faire du réalisme socialisme un aboutissement à la Résistance, l’origine de la création du journal. 

	André Masson n’est pas étranger à l’expression de l’acte de résistance, mais la création ne peut pas venir du collectif, mais au contraire d’un point de vue intime, puisque la liberté des traits du dessin vient de la notion de désir. C’est pourquoi la liberté est polyphonique dans les oeuvres de Masson : Libertaire comme dans ce \emph{Dessin}, et l’expression appuyée du corps féminin nu et la retranscription en mouvement d’un autre type de cri qui est celui de la révolte. Mais la nature même du dessin automatique et sa remontée vers l’inconscient ne permet pas de reposer sur une dimension collective propre au réalisme socialiste. Cependant, d’un point de vue intime ou collectif, André Masson comme le réalisme socialiste cherchent à retranscrire le ressenti du peuple avant même sa représentation. 

	C’est pourquoi la finalité réclamée par Aragon dans les dernières lignes de sa critique littéraire rejoint celle du Dessin d’André Masson et sa figuration comme ressenti du réel, et non plus son mimétisme : 

\begin{quote}
Un jour, messieurs les liquidateurs, la jeunesse de ce pays déchirera vos écrits avec ce mépris que donnent les grands enthousiasmes, ce mépris que ma génération eut pour vos prédécesseurs qui ignoraient Rimbaud au nom de Sully Prudhomme. 	
\end{quote}
 
	 En réactivant la vieille querelle de Rimbaud contre les Parnassiens, Aragon sous-entend que le réalisme socialiste est méprisé, du moins en partie en France, pour d’autres courants littéraires privilégiés qui occupent le devant de la scène. Mais on retrouve aussi dans cette défense du réalisme socialiste les prémices de la thèse défendue en 1959 dans \emph{Savoir aime}r à propos du travail de critique : Le réalisme socialiste a pour essence l’\enquote{enthousiasme}, autrement dit ce qu’Aragon exige aussi de la critique en général.  

	Or, avec son exigence de vitesse lorsqu’il dessine, André Masson recherche lui aussi un jaillissement de l’expression au sens le plus vif. Cet élément qui est sans doute l’élément le plus fondamental du texte est aussi le plus intimement proche de l’esthétique d’André Masson : \enquote{Il ne suffit pas de dire que le feu ne brûle pas quand c’est tout un monde qui s’embrase}. Cette phrase au présent gnomique aux allures de maxime est celle mise à part des autres et qui conclut l’article. Même si la passion qu’Aragon évoque comprend avec le \enquote{monde} la dimension collective théorisée précédemment, le lieu commun des flammes rejoint les métaphores astrales de Masson permanentes dans ses écrits : Si le réalisme socialise exige une forme collective, basée sur la tradition épique mais aussi plus onirique avec la référence aux troubadours et à la performance, le lien étroit entre le concept et l’art de Masson se fait au niveau d’une réception aux attentes analogues : Une certaine dévastation est attendue, un dérèglement des sens. Le lecteur d’une oeuvre réaliste socialiste et un spectateur d’une oeuvre de Masson doivent être à l’inverse de la passivité, ou de toute distanciation. 

	On peut d’ailleurs soupçonner Masson en dessinant à partir de l’intime pour produire ce mouvement conçu à partir de flux de pensées de parler en réalité non d’un homme mais de l’homme. Or, cette passion qui répond à deux projets différents mais qui est nécessaire dans la création répond aux mêmes exigences en terme de réception, et rejoint ainsi une conception spécifique de l’homme. L’image de la flamme est aussi et surtout le lieu commun du désir. Le désir est aussi l’essence même de la création de Masson. 

C’est pourquoi il ne s’agit pas en convoquant ces notions de soulèvement de passion de confondre l’art d’André Masson caractérisé par ce \emph{Dessin} dans l’idée que proclame Aragon sur le réalisme socialiste. D’abord, comme la négation à la tête du titre l’évoque, \emph{Non le réalisme socialisme n’est pas mort}, Aragon est dans une position de défense et de refus. On comprend d’ailleurs quel facteur décisif peut être l’appel aux passions pour justifier la continuité de son existence malgré cette période obscure pour le réalisme socialiste après le rapport de Khrouchtchev le 24 février 1956 où les crimes et déportations de Staline sont dénoncés pendant ce XXème Congrès de Moscou. 

% Voir Chronique du Bel Canto. 

	 Ainsi, en concluant par cette métaphore du désir, Aragon ne mentionne plus la passion autour du stalinisme, présente dans sa définition de 1952 lors de la remise du Prix Staline à André Stil. En revanche, il préserve une substance idéologique indépendante de la figure de Staline. Or, si l’on compare l’image de l’homme selon Aragon au nom de ce réalisme socialiste revisité et celui d’André Masson où du désir s’exprime une vision de l’homme, sur l’homme, on voit que de ces deux désirs aboutissent une conception idéologique proche et intime. Il n’est pas certain que les autres auteurs réalistes socialistes ou les dirigeants et militants communistes approuvent totalement la réorientation du réalisme socialiste d’Aragon, forcément plus personnelle à présent, et que ce socialisme se rapproche de la vision socialiste aussi d’André Masson sur les hommes. C’est pourquoi on peut se demander si l’origine de cette perception idéologique, créative et réceptive n’est pas due à cette conviction essentielle chez les deux hommes que l’idéologie politique naitrait de leur création, \enquote{et non pas l’inverse}. 

\section{Discours récurrents des \emph{Lettres françaises} autour d’André Masson (n°1104- du 4 au 10 novembre 1965]}