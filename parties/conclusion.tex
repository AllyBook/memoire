\chapter*{Conclusion} \markboth{Conclusion}{Conclusion}
Le fil du lyrisme révolutionnaire dans les numéros des \emph{Lettres françaises} entre Aragon et André Masson s'inscrit plus largement dans la réactualisation de l'amitié de ces deux hommes. Tous deux apparaissent comme les survivants des grands mouvements de leur jeunesse. Les années 1960, particulièrement, multiplient les hommages de leurs amis du groupe surréalsites disparus. Il resort de ces collaborations des croisements idéologiques et esthétique Ce n'est sans doute pas un hasard si le Aragon romancier et directeur de journal dans les années 1960 prolonge les recherches esthétiques et idéologique de ses jeunes années surréalsites dans une perspective qui se mêlent à celles de Masson. 

Les temps d'apparente perte de contact, ou tout du moins d'échanges plus discrets sont également évocateurs, et le rapport au lyrisme révolutionnaire de l'un et de l'autre est un exemple symbolique:la sensiblité de l'un et de l'autre pour la Commune de Paris l'illustre en profondeur, puisqu'elle constitue un moment historique qui influence directement le projet romanesque et journalsitique d'Aragon et l'art de Masson. Cependant, aussitôt ce point commun fort formulé, les distinctions politiques de l'un et de l'autre ne permettent pas la fusion complète autour du lyrisme révolutionnaire qu'incarne pourtant la Commune dans leur imaginaire. Pourtant, philosphiquement parlant, Aragon et André Masson ne font que rechercher cette liberté totale déjà entrevue dans leurs premières collaborations. Une même philisophie qui fait ce retrouver ces deux survivants de leur génération s'est donc bâtie sur des rapports politiques distincts, bien que tous deux se situent sur la partie gauche de l'échiquier politique, et que leurs distinctions n'empêchent pas le retour à un imaginaire commun autour du lyrisme révolutionnaire lui même traité différemment dans l'évolution des \emph{Lettres françaises}. Le lyrisme révolutionnaire incarne ainsi l'un des grands fils médiateurs de la politique éditoriale du journal.

Ainsi, dans cette ligne La révelation de cette proximité même bien après le surréalisme pour des formes et des causes de combats similaires  fait de ce journal le lieu de ces combats idéologiques et culturels. Le lyrisme révolutionnaire en demeure l'une des marques de fabrique. Il en resort que \emph{Les Lettres françaises} illustrent dans ces associations d'idées, qui relèvent du procédé revendiqué par Aragon dans ses romans, avec cette métaphore du fil à Tisser emploér dans \emph{Blanche ou l'oubli}. Aragon et Masson n'ont jamais pu tout à fait être en rupture contrairement à ce qu'ils auront vécu tous deux chacun de leur côté vis-à-vis de Breton, à plusieurs reprises pour Masson, mais que \emph{Les Lettres françaises} ont agi comme un lieu et un prétexte à la réactualsation de l'amitié  tout en révélant que l'un et l'autre même par des voies différentes n'ont jamais tout à fait quitté le fil du lyrimse révolutionnaire. 

En somme,cette métaphore du fil tissé emprunté par Aragon à Littré dans \emph{Blanche ou l'oubli} pour symboliser la conception romanesque à partir d'associations d'idées entremêlées se répercute dans la politique éditoriale des \emph{Lettres françaises}. C'est pourquoi prolonger le suivi du fil de ces associations d'idées comme l'offre une plateforme telle que l'\emph{Eman}. Cet espace numérique rassemble des numéros numérisés et offre ainsi grâce à ses outils de mise en correspondance la possibilité de remonter au plus près des obsessions politiques et esthétiques des numéros, et même de remonter jusqu'aux dialogues entre le journal et les romans d'Aragon. 