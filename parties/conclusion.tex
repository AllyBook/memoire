\chapter*{Conclusion} \markboth{Conclusion}{Conclusion}

L'aventure du lyrisme révolutionnaire dans les numéros des \emph{Lettres françaises} entre Aragon et André Masson s'inscrit plus largement dans la réactualisation de l'amitié de ces deux hommes, tous deux âgés et perçus mutuellemet comme les survivants des grands mouvements de leur jeunesse. Tandis que depuis les années 1960 se multiplient les hommages de leurs amis du groupe surréalsites disparus, l'un et l'autre affichent leur amitié, et avec elle leurs croisements idéologiques et esthétiques, principalement autour de collaborations d'hommages. La qualité d'hommage pour un tiers, un être cher en commun, n'omet en rien l'importance des retrouvailles de cex des hommes au destins croisés sur bien des aspects. A commencer par le passé sur le champ de bataille durant la 1ère Guerre Mondiale, mais dont déjà la différence de psote, donc d'expérience, de l'un et de l'autre représente quelles proximités Aragon et Masson connaissent dans leurs parcorus respectifs, tout en se ditinguant par une orientation idéologique différente. Les premières grandes collaborations entre Aragon et André Masson commencent dans la transition entre dada et le surréalisme pour Aragon, lorsqu'il rencontre les artistes de la rue Blomet, et que ses collaborations telles \emph{Le Con d'Irène  } en 1928 avec André Masson s'attèlent à la recherche de l'exepression la plus libre possible, exprimée par l'érotisme. Mais ce n'est sans doute pas un hasard si le Aragon romancier et directeur de journal dans les années 1960 prolonge les recherches esthétiques et idéologique de ses jeunes années surréalsites dans une perspective qui croise celles de Masson. Avec des collaborations d'hommage de sortes très diverses, de la mort de Georges Limbour, l'une des grandes figures médiatrices entre eux à l'origine de leur rencontre, jusqu'à l'hommage à Pablo Neruda lui bien vivant en raison de l'explosion de sa maison, sans compter les articles cette fois en l'honneur de Masson ou ceux d'Aragon et de Masson dont les associations d'idées s'entremêlent d'un numéro à l'autre, \emph{Les Lettres françaises} devient le lieu polyphonique de ce nouveau temps d'amitié. 

Les temps d'apparente perte de contact, ou tout du moins d'échanges plus discrets sont également évocateurs, et le rapport au lyrisme révolutionnaire de l'un et de l'autre en manifeste un exemple symbolique:la sensiblité de l'un et de l'autre pour la Commune de Paris l'illustre en profondeur, puisqu'elle constiute un moment historique qui influence directement le projet romanesque et journalsitique d'Aragon et l'art de Masson. Cependant, aussitôt ce point commun fort formulé, les distinctions politiques de l'un et de l'autre ne permet pas la fusion complète autour du lyrisme révolutionnaire qu'incarne pourtant la Commune dans leur imaginaire. La cause en revient à leur distinction idéologique, en particulier lorsqu'Aragon quitte le groupe surréaliste pour le parti communiste, tandis que Masson non seulement refuse ce rapport militant vis-à-vis d'un parti, et surtout considérant les thèses marxistes comme des formes de servitude de l'homme. Pourtant, philosphiquement parlant, Aragon et André Masson ne font que rechercher cette liberté totale déjà entrevue dans leurs premières collaborations. Une même philisophie qui fait ce retrouver ces deux survivants de leur génération s'est donc bâtie sur des rapports politiques distincts, bien que tous deux se situent sur la partie gauche de l'échiquier politique, et que leurs distinctions n'empêchent pas le retour à un imaginaire commun autour du lyrisme révolutionnaire lui même traité différemment dans l'évolution des \emph{Lettres françaises} : d'abord vis-và-vis du contexte de la guerre froide depuis la fin des années 40 et dans les années 1950, où le lyrisme révolutionnaire s'inscrit dans la perspective du réalisme socialiste. Dans les années 1960, le lyrisme révolutionnaire est l'un des éléments qui nourrit les nouvelles orientations romanesques autour de la mise en doute de l'identité. André Masson valorise cette question sur les articles qu'il loue dans ses articles. Toujours est-il que, malgré cette évolution, les numéros en l"honneur des anniversaires de la Commune reflètent la marque de fabrique des \emph{Lettres françaies} comme journal qui naît pendant la Résistance et forge sa politique éditoriale avec le fil directeur de la Résistance. Le lyrisme révolutionnaire opère ainsi un fil médiateur de la ligne éditoriale.

Ainsi, dans cette ligne éditoriale, le lien entre Aragon et André Masson révèle par son procédé d'associations di'idées des similarités très fortes malgré le parcours disticnt des deux hommes, en particulier dans leurs articles de critiques, où tous deux défendent le parti pris du lyrisme et de la subjectivité. Peut-être ces analogies sont-elles dues aux proximités philosophiques du vertige d'Aragon à la fête chez André Masson. Quelque chose de l'ordre de la synésthésie, du retournement de sens émerge dans les deux cas. Avec cette tension conflictuelle entre le désir de liberté et celui de débordement intérieur jusqu'à ses impacts physiques. 


