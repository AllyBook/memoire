\chapter*{Conclusion} \markboth{Conclusion}{Conclusion}
Le fil du lyrisme révolutionnaire dans les numéros des \emph{Lettres françaises} entre Aragon et André Masson s'inscrit plus largement dans la réactualisation de l'amitié de ces deux hommes. Tous deux apparaissent comme les survivants des grands mouvements de leur jeunesse. Les années 1960, particulièrement, multiplient les hommages de leurs amis du groupe surréalsites disparus. Il resort de ces collaborations des croisements idéologiques et esthétique Les premières grandes collaborations entre Aragon et André Masson commencent dans la transition entre dada et le surréalisme pour Aragon, lorsqu'il rencontre les artistes de la rue Blomet, et que ses collaborations telles \emph{Le Con d'Irène  } en 1928 avec André Masson s'attèlent à la recherche de l'exepression la plus libre possible, exprimée par l'érotisme. Mais ce n'est sans doute pas un hasard si le Aragon romancier et directeur de journal dans les années 1960 prolonge les recherches esthétiques et idéologique de ses jeunes années surréalsites dans une perspective qui croise celles de Masson. Les associations d'idées s'entremêlent d'un numéro à l'autre, \emph{Les Lettres françaises} devient le lieu polyphonique de ce nouveau temps d'amitié. 

Les temps d'apparente perte de contact, ou tout du moins d'échanges plus discrets sont également évocateurs, et le rapport au lyrisme révolutionnaire de l'un et de l'autre est un exemple symbolique:la sensiblité de l'un et de l'autre pour la Commune de Paris l'illustre en profondeur, puisqu'elle constitue un moment historique qui influence directement le projet romanesque et journalsitique d'Aragon et l'art de Masson. Cependant, aussitôt ce point commun fort formulé, les distinctions politiques de l'un et de l'autre ne permet pas la fusion complète autour du lyrisme révolutionnaire qu'incarne pourtant la Commune dans leur imaginaire. La cause en revient à leur distinction idéologique, en particulier lorsqu'Aragon quitte le groupe surréaliste pour le parti communiste, tandis que Masson non seulement refuse ce rapport militant vis-à-vis d'un parti, et surtout considérant les thèses marxistes comme des formes de servitude de l'homme. Pourtant, philosphiquement parlant, Aragon et André Masson ne font que rechercher cette liberté totale déjà entrevue dans leurs premières collaborations. Une même philisophie qui fait ce retrouver ces deux survivants de leur génération s'est donc bâtie sur des rapports politiques distincts, bien que tous deux se situent sur la partie gauche de l'échiquier politique, et que leurs distinctions n'empêchent pas le retour à un imaginaire commun autour du lyrisme révolutionnaire lui même traité différemment dans l'évolution des \emph{Lettres françaises} : d'abord vis-và-vis du contexte de la guerre froide depuis la fin des années 40 et dans les années 1950, où le lyrisme révolutionnaire s'inscrit dans la perspective du réalisme socialiste. Dans les années 1960, le lyrisme révolutionnaire est l'un des éléments qui nourrit les nouvelles orientations romanesques autour de la mise en doute de l'identité. André Masson valorise cette question sur les articles qu'il loue dans ses articles. Toujours est-il que, malgré cette évolution, les numéros en l"honneur des anniversaires de la Commune reflètent la marque de fabrique des \emph{Lettres françaies} comme journal qui naît pendant la Résistance et forge sa politique éditoriale avec le fil directeur de la Résistance. Le lyrisme révolutionnaire opère ainsi un fil médiateur de la ligne éditoriale.

Ainsi, dans cette ligne éditoriale, le lien entre Aragon et André Masson révèle grâce aventure procédé d'associations di'idées des similarités très fortes malgré le parcours disticnt des deux hommes, en particulier dans leurs articles de critiques, où tous deux défendent le parti pris du lyrisme et de la subjectivité. Et même, plus encore que la subjectivité, de l'enthousiasme du critique, à partir duquel le lyrisme naît. Le lyrisme de ces critiques vient donc par essence de l'engagement, c'est-à-dire du parti pris du critique. Peut-être ces analogies dans leurs marques de discours sont-elles dues aux proximités philosophiques du vertige d'Aragon à la fête chez André Masson. Quelque chose de l'ordre de la synésthésie, du retournement de sens émerge dans les deux cas. Avec cette tension conflictuelle entre le désir de liberté et celui de débordement intérieur jusqu'à ses impacts physiques. Ces résurgences idéologiques et esthétiques qui n'ont jamais vraiment quitté l'un et l'autre mais fusionnent en particulier dans les années 1960 s'expriment paradoxalement en saluant ces figures qui ont été médiatrices entre eux, d'abord celles du surréalisme telles que Breton et Limbour. Mais aussi d'autres toutes aussi fortes moins reconnues à première vue comme pouvant être une passerelle entre Aragon et André Masson, mais dont la figure de Pablo Neruda, celle de Jean-Louis Barrault, expriment de tout aussi profonds combats poltiiques et culturels bien après la période du surréalisme. La révelation de cette proximité même bien après le surréalisme pour des formes et des causes de combats similaires avec les mêmes acteurs entre établie par \emph{Les Lettres françaises} fait de ce journal le lieu de ces combats idéologiques et culturels. Le lyrisme révolutionnaire en demeure l'une des marques de fabrique, en évolution par essence comme la forme du mot \enquote{Réssitance} recherchée par le journal selon les époques. Il en resort que \emph{Les Lettres françaises} révèlent dans ces associations d'idées, qui relèvent du procédé revendiqué par Aragon dans ses romans, mettent en lumière non seulement ces proximités idéologiques et esthétiques qui font qu'Aragon et Masson n'ont jamais pu tout à fait être en rupture contrairement à ce qu'ils auront vécu tous deux chacun de leur côté vis-à-vis de Breton, à plusieurs reprises pour Masson, mais que \emph{Les Lettres françaises} ont agi comme un lieu et un prétexte à la réactualsation de l'amitié et la collbaoration entre Aragon André Masson, tout en révélant que l'un et l'autre même par des voies différentes pour exprimer n'ont jamais tout à fait quitté le fil du lyrimse révolutionnaire dont il devient en particulier dans les numéros de 1960 l'un des ponts de correspondances entre eux. 


