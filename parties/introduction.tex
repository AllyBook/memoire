Connu avant tout comme le grand poète de la Résistance, son passé de jeune surréaliste et sa fidélité par la suite au parti communiste, l'écrivain Louis Aragon est moins étudié du point de vue de ses romans et de ses écrits sur l'art. En outre, sa riche expérience journalistique est quasiment passée sous silence, bien que largement évoquée dans le premier tome Aragon \footcite[]{cahiers} paru dans Les Cahiers et relatée dans la biographie Aragon \footcite[]{biographie}  de Philippe Forest. Paradoxalement, Philippe Forest rappellerait à propos des \emph{Lettres françaises}, le dernier plus important journal auquel Aragon a collaboré puis dirigé, que celui-ci voit le jour justement dans ce contexte de Résistance comme geste subversif vis-à-vis de Vichy:
\begin{quote} 
Les intellectuels sont invités à se regrouper autour d'un réseau unifié - Ce sera le Comité national des écrivains - et à se doter des moyens qui permettront de diffuser le mot d'ordre de la Résistance: il s'agira d'un journal, Les \emph{Lettres françaises}\footcite[]{biographie}.
\end{quote}