\chapter*{Introduction}
\addcontentsline{toc}{chapter}{Introduction}
Connu avant tout comme le grand poète de la Résistance, son passé de jeune surréaliste et sa fidélité par la suite au parti communiste, l'écrivain Louis Aragon est moins étudié du point de vue de ses romans et de ses écrits sur l'art. En outre, sa riche expérience journalistique est quasiment passée sous silence, bien que largement évoquée dans le premier tome Aragon \footcite[]{cahiers} paru dans Les Cahiers et relatée dans la biographie Aragon \footcite[]{biographie}  de Philippe Forest. Paradoxalement, Philippe Forest rappellerait à propos des \emph{Lettres françaises}, le dernier plus important journal auquel Aragon a collaboré puis dirigé, que celui-ci voit le jour justement dans ce contexte de Résistance comme geste subversif vis-à-vis de Vichy:
\begin{quote} 
Les intellectuels sont invités à se regrouper autour d'un réseau unifié - Ce sera le Comité national des écrivains - et à se doter des moyens qui permettront de diffuser le mot d'ordre de la Résistance: il s'agira d'un journal, \emph{Les Lettres françaises}.\footcite[]{biographie}
\end{quote}
On peut donc s'interroger sur cette distinction dans la mémoire collective entre la poésie de résistance renommée et les écrits sur  moins reconnus, notamment ceux de la revue \emph{Les Lettres françaises}, alors que ces derniers nourrissent à la même période une forme de résistance analogue. C'est pourquoi la réhabilitation de la figure d'Aragon, dont les idées artistiques et sociétales ont la particularité de se faire aujourd'hui encore reflets d'une réalité contemporaine, peut idéalement se retrouver par fragments grâce aux numéros de la revue \emph{Les Lettres françaises}.


Qu'est-ce que \emph{Les Lettres françaises}? C'est d'abord une revue fondée en 1941 par les écrivains Jacques Decour, communiste et Jean Paulhan, déjà journaliste-résistant, rassemblés par Aragon. On ne manquera pas donc pas de relever le rôle fondamental d'Aragon en tant que médiateur dans la création d'une revue réfléchie comme un «front littéraire». Par la suite, aucun bandeau d'un numéro du journal n'omettra de préciser le nom de ses deux fondateurs, avec la fameuse mention: «Fondateurs: Jacques Decour (fusillé par les nazis) et Jean Paulhan.» Ainsi, lorsqu'Aragon prend lui-même la direction de 1953 à 1972 de cette revue financée par le parti communiste, le bandeau vert indique d'ores et déjà sa volonté de prolonger le rôle tant culturel que résistant des \emph{Lettres françaises}.Comment le rôle de front littéraire sous la résistance évolue-t-il après la guerre sous la direction d'Aragon? Par essence, cette revue d'art a pour marque de fabrique l'entremêlement de l'oeuvre d'art et du propos politique. Bien que l‘aspect politique ne puisse que croiser une idéologie communiste, le projet de démocratisation de l'oeuvre d'art voulue à la portée du peuple demeure un enjeu qui peut paraître aujourd'hui encore défendable, ou du moins qui peut être débattu aujourd'hui encore. Cette filiation sous la direction d'Aragon aboutit grâce à la correspondance des illustrations massives dans la revue et des textes qui leur font écho, comme le précise Julie Morisson dans son article \emph{L'écran Journal} paru dans Les Cahiers: «\emph{Les Lettres françaises} se fait musée et transforme le lecteur en spectateur»\footcite[]{cahiers}. Ainsi, la mise en page est l'un des fondements de la politique éditoriale du journal. Elle évoque le principe artistique du découpage où le message non-verbal qu'est l'image est une parole. En somme, il s'agit d'une gigantesque exposition qui relate une actualité, une vision du monde, par le biais du langage, des oeuvres d'arts.