\chapter*{Introduction} \markboth{Introduction}{Introduction}
\addcontentsline{toc}{chapter}{Introduction}

Connu avant tout comme le grand poète de la Résistance, son passé de jeune surréaliste et sa fidélité par la suite au parti communiste, l'écrivain Louis Aragon écrit tout autant des romans et des écrits sur l'art. Sa riche expérience journalistique est  décrite dans le premier tome Aragon \footcite{cahiers} paru dans Les \emph{Cahiers} et relatée dans la biographie Aragon \footcite[]{biographie} de Philippe Forest. Philippe Forest rappelle à propos des \emph{Lettres françaises}, le dernier plus important journal auquel Aragon a collaboré puis dirigé, que celui-ci voit le jour justement dans ce contexte de Résistance comme geste subversif vis-à-vis de Vichy:
\begin{quote} 
Les intellectuels sont invités à se regrouper autour d'un réseau unifié --- Ce sera le Comité national des écrivains --- et à se doter des moyens qui permettront de diffuser le mot d'ordre de la Résistance: il s'agira d'un journal, \emph{Les Lettres françaises}.\footcite[p492]{biographie}
\end{quote}
On peut donc s'interroger sur cette distinction dans la mémoire collective entre la poésie de résistance renommée et les écrits sur l'art. Ces derniers offrent une large place à la poésie d'Aragon, notamment ceux de la revue \emph{Les Lettres françaises}. Mais c'est aussi le cas antérieurement au sein du journal \emph{Ce soir} dans lequel Aragon écrit dès 1937, \emph{L'Humanité} en 1933 ou encore \emph{Commune} en 1934. Les écrits sur l'art comme les poèmes d'Aragon ont ainsi pour point commun l'illustration d'une part importante de la prodction journalsitique d'Aragon, et ce bien avant la période de la Réssitance. Le recueil \emph{Crève C\oe{}ur}de 1941 emblématique de la poésie de résistance précède de quelques mois la naissance des \emph{Lettres françaises}le septembre 1942. C'est pourquoi la réhabilitation de la figure d'Aragon peut idéalement se retrouver par fragments grâce aux numéros de la revue \emph{Les Lettres françaises}. Sans compter que l'expérience de directeur de journal avant celle de 1953 est loin d'être la première : il dirige \emph{Ce soir} jusqu'à sa demission en 1947,\emph{Europe}, originellement crée par Romain Rolland. Aragon porte peut-être ainsi l'héritage de Romain Rolland lorsqu'il mêle son expérience journalistique à sa réflexion romanesque.


Qu'est-ce que \emph{Les Lettres françaises}? C'est d'abord une revue fondée en 1941 par les écrivains Jacques Decour, communiste et Jean Paulhan, déjà journaliste-résistant, rassemblés par Aragon. On ne manquera pas donc pas de relever le rôle fondamental d'Aragon en tant que médiateur dans la création d'une revue réfléchie comme un \enquote{front littéraire}. Par la suite, aucun bandeau d'un numéro du journal n'omettra de préciser le nom de ses deux fondateurs, avec la fameuse mention: \enquote{Fondateurs: Jacques Decour (fusillé par les nazis) et Jean Paulhan.} Ainsi, lorsqu'Aragon prend lui-même la direction de 1953 à 1972 de cette revue financée par le parti communiste, le bandeau vert indique d'ores et déjà sa volonté de prolonger le rôle tant culturel que résistant des \emph{Lettres françaises}. Comment l'aspect du front littéraire sous la résistance évolue-t-il après la guerre sous la direction d'Aragon? Par essence, cette revue d'art a pour marque de fabrique l'entremêlement de l'\oe{}uvre d'art et du propos politique. Bien que l‘aspect politique ne puisse que croiser une idéologie communiste, le projet de démocratisation de l'\oe{}uvre d'art voulue à la portée du peuple demeure un enjeu qui peut paraître aujourd'hui encore défendable, ou du moins qui peut être débattu aujourd'hui encore. Cette filiation sous la direction d'Aragon aboutit grâce à la correspondance des illustrations massives dans la revue et des textes qui leur font écho, comme le précise Julie Morisson dans son article \emph{L'écran Journal} paru dans Les Cahiers: \enquote{\emph{Les Lettres françaises} se fait musée et transforme le lecteur en spectateur.\footcite[p. 169--172]{cahiers}} Ainsi, la mise en page est l'un des fondements de la politique éditoriale du journal. Elle évoque le principe artistique du découpage où le message non-verbal qu'est l'image est une parole. En somme, il s'agit d'une gigantesque exposition qui relate une actualité, une vision du monde, par le biais du langage, des \oe{}uvres d'arts.

Ce projet a la particularité de passer par un mouvement graphique constant comme reflet d’une parole politique. Les allusions des articles et illustrations à un fil conducteur commun au numéro, voire sur plusieurs, comme un prolongement de réflexion apportent au \emph{Lettres françaises }tout un réseau d'association d'idées.. Néanmoins, on constate également que l’un des grands représentants de ce mouvement graphique tant dans la mise en page de la revue que dans sa dimension politique, n’est autre que le peintre, sculpteur et dessinateur et ancien surréaliste André Masson. En quoi André Masson est-il un artiste emblématique des valeurs défendues par \emph{Les Lettres françaises}? En partie pour l’intérêt esthétique commun à l’écrivain et au peintre, c’est-à-dire le tracé originel, la genèse de l’\oe{}uvre. D’autre part, André Masson qui reçoit dans les premiers temps de la revue les numéros en Amérique où il est exilé depuis 1941, décrit celle-ci comme la dernière source d’espérance en l’humanité sous les tensions du fascisme en Europe en 1942: \enquote{je recevais les \emph{Lettres françaises} (la seule revue à cette heure qui nous rappelle que quelque fois les Français savent penser et écrire).}\footcite[p478]{anneessurrealistes}. Dès les premiers numéros, André Masson est déjà un fidèle lecteur de l’autre côté de l’Atlantique, puisque \emph{Les Lettres françaises} lui parviennent par voie postale, assez facilement pour qu’il la recommande à des proches éloignés géographiquement: 
\begin{quote}
Encore une demande: puis-je vous demander de me faire le grand plaisir d’envoyer \enquote{Lettres françaises} à mon parent: Sergent Robert Piel. 3\ieme{} Compagnie Bataillon de Marche \No{}6; Moyen Congo. Afrique françaises libre. Le brave garçon est sevré de lectures et me demande des revues.\footcite[]{anneessurrealistes}
\end{quote}
Il est d’ailleurs révélateur de relever l’analogie entre l’esthétique de Masson, passionné par le geste originel, le mouvement du tracé, jusque dans le thème de ses \oe{}uvres avec sa série de dessins \emph{Mythologies}, et la propre esthétique d’écriture d’Aragon. Considéré comme l’un des grands acteurs du dessins automatique dans le groupe surréaliste, Masson reste longtemps lié au groupe, lui-même faisant parti du fameux groupe d’artistes dans les ateliers de la rue Blomet. 

La relation entre Aragon, qui connaît les artistes de la rue Blomet et Masson est forte mais discrète en sources documentaires: les deux hommes se rencontrent au lendemain de la guerre. Aucune correspondance n’exprime clairement une interaction entre eux. En revanche, les allusions de l’un à l’autre sont assez présentes pour rendre d’autant plus forte le lien qui unit les deux hommes. Masson ne manque pas de préciser le rôle joué par Aragon comme médiateur avec le mécène Jacques Doucet afin que Masson obtienne d’entrer dans la collection de ce dernier. Masson est l’illustrateur du récit érotique \emph{Le con d'Irène} présent dans les textes fragmentés de \emph{La Défense de l’Infini} (1928). Sans oublier l’hommage d’Aragon à Masson dans \emph{Le Paysan de Paris} de 1924. 

D’autre part, l’écrivain et l’artiste ont tous les deux traversés une grande partie du \siecle{XX}: la date exacte de naissance d’Aragon reste un mystère mais se fixe le 3 octobre 1897 jusqu’à celle de son décès, elle très précise, le 24 décembre 1982. D’une longévité plus grande encore, André Masson est son aîné d’un an seulement né le 4 janvier 1896, décédé le 28 octobre 1897. En somme, cent ans après la naissance d’Aragon. On a donc affaire à deux hommes aux formes d’expressions différentes, mais aux esthétiques analogues, qui ont littéralement vécu le vingtième siècle, lui même riche d’Histoire, ses deux guerres mondiales, et d’évolutions sociales. Avant même leur adhésion au surréalisme, André Masson et Louis Aragon partagent à des postes différents l’expérience de la 1ère Guerre Mondiale. Et, en particulier, l’épisode du Chemin des Dames où Masson est gravement blessé au bras suite aux ordres abusifs et suicidaires d’un commandant, au point de quitter le champ de bataille pour être balloté d’hôpital en hôpital. Aragon, en qualité d'auxilliaire médoical, vit la guerre d’un autre poste et continue de la vivre après l’armistice ce qui le rend absent de la sphère littéraire parisienne et des premiers travaux expérimentaux telle que l’écriture automatique dans le \emph{Champ magnétique} de Breton et de Soupault en 1919. Aux premiers temps des aventures dada, lui reçoit à distance les revues sur le champ de bataille, toujours mobilisé pendant que le reste de ses amis cherchent à composer un groupe littéraire plus transgressif que les offres proposées, c'est-à-dire en marge des formes académiques. 

Le traumatisme de cette guerre hante les deux hommes mêmes des années plus tard à des âges avancés, comme l’attestent les écrits d’Aragon avec les premiers poèmes du recueil \emph{Roman inachevé} de 1956. Ou encore  le discours fataliste derrière la scène euphorique de l’épilogue du roman \emph{Les cloches de Bâle}. L’art tourmenté et désillusionné de Masson rejoint cette perspective. En outre, ces deux fameux membres du groupe surréaliste quittent à peu près à la même période le mouvement: Aragon en 1932, Masson en 1929, bien que ce dernier ne rompe vraiment son amitié avec Breton qu’en 1943 après une brève seconde période surréaliste, et surtout après leur exil ensemble en Amérique en 1941. On pourrait relever une certaine analogie entre le rapport d’Aragon et Masson vis-à-vis du projet surréaliste: là où Aragon pratique peu l’écriture automatique, Masson, lui, bien que représentant du dessin automatique, le pratique selon une perspective physique, organique à l’inverse des autres artistes qui iront vers une dimension plus onirique. De plus, au moment de la grande lecture par les surréalistes du \emph{Paysan de Paris}, Philippe Forest rappelle la distinction des visions d’Aragon et de Breton sur la signification du mouvement: 

\begin{quote}
Et quand Aragon, par la bouche d’une oute figure allégorique, \enquote{l’imagination}, se fait le théoricien du surréalisme nouveau, la définition qu’il en donne porte exclusivement l’accent sur l’image entendue comme vecteur essentiel de la création littéraire sans qu’intervienne aucune véritable référence à l’automatisme promu par Breton.\footcite[]{biographie}
\end{quote}

Quant aux \oe{}uvres de Masson, les surréalistes s’intéresseront donc plus à ses toiles et \oe{}uvres les plus \enquote{achevées} selon Masson plutôt qu'à ses dessins les plus expérimentaux. Plusieurs sources appuient ce constat : 

\begin{quote}
Masson s’étonne que ses dessins automatiques, bien que publiés régulièrement dans la Révolution surréaliste, soient négligés par les membres du groupe, qui ne cherchent ni à les conserver ni à les acheter --- ils ne s’intéressent qu’aux tableaux et aux dessins que Masson, quant à lui, juge plus \enquote{laborieux --- plus arrêtés} --- et donc \enquote{moins surréalistes}.\footcite[p. 28]{noel}
\end{quote}
Bernard Noël emprunte des termes puisés d’écrits de Masson, dans lesquels le peintre non seulement déplore ce constat, mais en plus l’associe à sa rupture à venir avec le groupe surréaliste : 
\begin{quote}
Si j’en reviens à mes premières tentatives d’automatisme graphique \textelp{} je dois aussitôt ajouter que ce n’était pas ces manifestations (naïvement je les croyais vraiment orthodoxes) qui retenaient l’attention de mes compagnons de route. Leur préférence allait à mes tableaux ou dessins  plus \enquote{laborieux} –-- plus arrêtés. Je ne ruminai pas trop ce paradoxe ; pour diverses raisons je quittai le groupe pour la première fois. C’était en 1929.\footcite[p. 35]{anneessurrealistes}
\end{quote}
Or, avant même que les écrits d’Aragon et les \oe{}uvres de Masson ne se croisent dans \emph{Les Lettres françaises}, on retrouve une esthétique commune centrée sur une conception \textbf{révolutionnaire}, c’est-à-dire un mouvement insurrectionnel au c\oe{}ur des \oe{}uvres des deux hommes: le mot \enquote{révolution} est martelé dans la série de romans réalistes-socialistes d’Aragon, \emph{Le monde réel}. Le thème révolutionnaire chez Aragon ne se limitera donc pas à l’expérience de la révolution surréaliste, bien que son recueil de 1926 \emph{Le mouvement perpétuel}illustre déjà la volonté d’une agitation, tant poétique que sociale. Chez Masson, la révolution est une nécessité de sa condition artistique rappelée dans l’essai de Bernard Noël: \enquote{L’une des rares interventions écrites que fit Masson dans La Révolution surréaliste est cette phrase , que la typographie met en évidence, comme un slogan: \enquote{IL FAUT SE FAIRE UNE IDÉE PHYSIQUE DE LA RÉVOLUTION}.\footcite[p. 28]{noel}}

Ainsi, la révolution ne sera pas chez Masson uniquement représentée, mais bien l’énergie même de l’\oe{}uvre, sa vitalité organique. De plus, les correspondances  entre 1916 et 1942 de Masson abordent massivement cette idéologie  révolutionnaire comme indispensable au geste artistique: \enquote{Que cette agitation cesse et je ne serais plus révolutionnaire et je ne peindrais plus\footcite[p. 102]{anneessurrealistes}} affirme Masson dans l’une de ses correspondances. Non seulement l’imagerie révolutionnaire rassemble les deux hommes, mais en plus chacun se rencontre sur le sentiment recherché dans leur \oe{}uvre: le terme \enquote{vertige}, qui reprend métaphoriquement parlant un retournement des sens tel que ne peut qu’en susciter qu’une révolution même intérieure, réceptive. 

Mais, il faut souligner que, bien avant leur relation très distincte mais passionnelle à l’égard du journal culturel \emph{Les Lettres françaises}, les deux hommes connaissent des expériences journalistiques, mais dans des revues différentes. André Masson n’est pas de la partie dadaïste, il ne fait donc pas, au contraire d’Aragon, ses débuts journalistiques avec des revues dadaïstes telles que \emph{Dada} et celle qui figure la transition vers le Surréalisme, \emph{Littérature}. En outre, il est intéressant de constater que dans la revue d’actualité théâtrale où il collabore par la suite, \emph{Paris-Journal}, sa première réelle expérience de journaliste, Aragon ambitionne de donner à la culture une place de poids dans la société par le biais du journal. Projet qui ne quitte pas Aragon jusqu’à sa prise de fonction comme directeur des \emph{Lettres françaises} en 1953. Sans compter que les deux hommes écrivent collaborent tous deux dans les revues surréalistes. (\emph{La Révolution surréaliste}, \emph{La revue européenne}) André Masson, lui, connaît aussi des expériences journalistiques mais, comme Aragon, les plus marquantes se déroulent probablement après la période surréaliste: il fonde en 1934 avec Georges Bataille les revues \emph{Acéphale et Minotaure}. Le titre de la revue \emph{Acéphale} est illustré en première page du fameux personnage sans tête d’André Masson. Plus qu'une anecdote, le personnage du Minotaure sans tête à la une de la revue consitue une marque de fabrique, d'abord de collaboration entre Bataille et Masson, mais surtout comme signature esthétique qui le caractérisera en partie bien après les années 30. Preuve en est l'\oe{}uvre de Masson \emph{La mémoire du monde} dans lequel le peintre y répetorie les temps forts de son existence et les sujets picturaux les plus  représentatifs de sa vision du monde, la trace qu'il laisse sur celui-ci déjà âge de 78 ans. Or, dans son chapitre \emph{Mythologie personnelle}\footcite[p124]{memoiremonde}, Masson publie la fameuse Une du numéro d'\emph{Acéphale}
 sur Dionysos, et laisse l'impression qu'à long terme cette une est devenue un mythe dans l'histoire personnelle de Masson tout autant que les figures mythologiques présentées. Ce tournant artistique permis par la création de ce numéro mène d'ailleurs au prolongement de cette aventure : 
 
 \begin{quote}
 Avec Dionysos, ce fut le Minotaure et tout ce qui entoure le mythe du Labyrinthe qui nous devint familier, au point que nous l'emportâmes tous deux, dans le choix du titre d'une revue nouvelle : \emph{Minotaure}, à l'usage tout d'abord des dissidents du Surréalisme.\footcite[p130]{memoiremonde}\end{quote}
 
 Ainsi, comme pour Aragon, l’expérience journalistique nourrit la réflexion créatrice. Néanmoins, les thèmes les plus abondants chez Masson tels que les figures mythologiques et l'érotisme poursuivent logiquement ce qui émergeait de son passé surréaliste, en particulier ses illustrations en 1928 pour la nouvelle \emph{Le Con d'Irène} d'Aragon. 

C’est pourquoi notre réflexion peut naturellement se reporter sur le croisement des \oe{}uvres de Masson devenues illustrations et des écrits d’Aragon devenus articles au sein des \emph{Lettres françaises}, notamment entre 1953 et 1972 lorsqu’Aragon prend la direction du journal, toujours financé par le parti communiste. Comment l’esthétique révolutionnaire commune aux deux hommes peut-elle s’infiltrer dans le quotidien des gens? Si le journal est financé par le parti communiste, ce n’est pas en tant que revue communiste que \emph{Les Lettres françaises} est désignée, mais bien comme une revue d’art, des arts. Nous pouvons ainsi nous interroger sur le rôle phare du lyrisme révolutionnaire, dans les illustrations de Masson d’une part, les articles d’Aragon d’autre part, mais également de tous les articles et agencements de ce dernier en tant que directeur pour interpeller l’\oe{}il du lecteur tant physiquement que politiquement. En fait, c’est une correspondance des \oe{}uvres de Masson au sein de la revue d’Aragon mêlées à une orientation politique que va engendrer le lyrisme révolutionnaire. Qu’est-ce que le lyrisme révolutionnaire? C’est la poétique de la révolution, la part de passion romantique qui s’oppose à toute violence présupposée du geste révolutionnaire. C’est l’apparente contradiction entre l’agitation de Masson et son contrôle, ainsi que l’imaginaire d’Aragon qui conduit au mouvement de révolte et d’agitation vers la création d’une transformation sociale radicale de la société. De la présence de Masson dans \emph{Les Lettres françaises} qui conduit à l’imaginaire communard d’Aragon et d’André Masson pour aboutir au lyrisme révolutionnaire dans la politique éditoriale des \emph{Lettres françaises}, les croisements et la collaboration des deux hommes révèlent un entremêlement esthétique vers une convergence idéologique. 